% !TEX encoding = UTF-8 Unicode
\documentclass{pclass}
\usepackage[utf8]{inputenc} % para linux y mac 
%\usepackage[latin1]{inputenc} % para windows
  

%DIFERENTES TIPOS DE LETRA
%\usepackage{palatino}
\usepackage{times}   

\usepackage{eurosym} % Simbolo €
\usepackage{csquotes} % \textquote{} con comillas angulares
\usepackage{enumerate}

\begin{document}
\tipo{Grado}
\titulopro{TITULO DEL PROYECTO} % TODO qué título debería tener el proyecto?
\tutor{Pablo Neira Ayuso}
\departamento{Lenguajes y Sistemas Informáticos}
\autores{(ponente): Andrés Durán Terrero}{{\ }}   % Un autor
\dia{1 de marzo de 2017 (v.0.91)} % TODO cambiar al entregar
\titulacion{GIS}

\hacerportada

	\makeatletter
\renewcommand*\l@section{\@dottedtocline{1}{0em}{2.5em}}
\renewcommand*\l@subsection{\@dottedtocline{2}{1.5em}{3.2em}}
\renewcommand*\l@subsubsection{\@dottedtocline{3}{4.3em}{3.2em}}
\makeatother

\renewcommand{\frontmatter}{\pagenumbering{Roman}}
\frontmatter
        
    \cdpchapter{Resumen}

El resumen del Trabajo fin de Grado consiste, como su propio nombre indica, es un resumen de la memoria en formato apropiado 
para ser indexado en las bases de datos bibliotecarias. No debe ocupar m\'as de una carilla de texto
y en ella hay que exponer en pocas palabras la finalidad y objetivos del trabajo, 
as{\'\i} como las aportaciones realizadas. En general,
no incluir\'a figuras, cuadros ni referencias bibliogr\'aficas.

El resumen es obligatorio en espa{\~n}ol para todos los TfG. Es opcional incluir un \emph{Abstract} (resumen en ingl\'es)
en una hoja separada. Debe ser traducci\'on dle correspodiente resumen en español. En los TfG presentados en ingl\'es,
la inclusi\'on del \emph{abstract} es tmabi\'en obligatoria, as{\'\i} como en los TfG correspondientes a la titulaci\'on
de Ingenier{\'\i}a de la Salud.

El resumen de este documento es el siguiente:

Este trabajo pretende ser una gu{\'\i}a para uniformar los formatos de las memorias de los Trabajos fin de Grado de las titulaciones: 
\begin{itemize}
\item Grado en Ingenier{\'\i}a Inform\'atica --- Ingenier{\'\i}a de Computadores
\item Grado en Ingenier{\'\i}a Inform\'atica --- Ingenier{\'\i}a del Software
\item Grado en Ingenier{\'\i}a Inform\'atica --- Tecnolog{\'\i}as Inform\'aticas
\end{itemize}

de la Escuela T\'ecnica Superior de Ingenier{\'\i}a Inform\'atica de la Universidad de Sevilla.

Al mismo tiempo, se pretende que el documento sea un ejemplo de la realizaci\'on de un memoria de Trabajo fin de Grado. 
Debido a ello, hemos estructurado el documento en cap{\'\i}tulos e incluido diversos {\'\i}ndices y bibliograf{\'\i}a,
aunque obviamente no hubiera sido necesario.



    \cdpchapter{Agradecimientos}

A nuestros alumnos y a nuestras alumnas.	

	\tableofcontents % Índice de contenidos
 	\listoftables % Índice de cuadros
 	\listoffigures % Índice de figuras
 	\lstlistoflistings %Índice de códigos

 \mainmatter  
    
     % !TEX root = ../proyect.tex

\chapter{Introducción}\label{cap1} % TODO: Como llamo a este capítulo?

\section{Introducción al problema}\label{sec:intro}

Los productores locales suelen enfrentarse a importantes dificultades a la hora de vender sus productos. En muchos casos, están limitados por el número de compradores en su área local, o se ven obligados a vender sus productos a través de intermediarios, que se llevan una parte importante de sus beneficios. Esto puede dificultar que los productores locales obtengan un precio justo por sus productos y lleguen a nuevos clientes.

Además, muchos productores locales no tienen acceso a los recursos y la infraestructura necesarios para comercializar sus productos con eficacia. Puede que no tengan una fuerte presencia en Internet o la capacidad de llegar a clientes potenciales a través de los canales tradicionales. Como resultado, a menudo se ven obligados a depender de los mercados o supermercados locales, lo que puede limitar su capacidad para llegar a nuevos clientes y hacer crecer su negocio.

Esto supone un reto importante para los productores locales que quieren vender sus productos directamente a los consumidores. Les resulta difícil encontrar compradores interesados en sus productos y que se encuentren cerca. Al mismo tiempo, los compradores potenciales pueden tener problemas para encontrar a los productores locales y puede que no conozcan los productos disponibles en su zona.

\section{Descripción de la solución propuesta}\label{sec:descripcion_solucion}

Ante estos retos, es necesaria una plataforma que facilite a los productores locales la venta directa de sus productos a los consumidores. Una plataforma que ofrezca una forma sencilla, cómoda y eficaz de conectar a productores y compradores, sin necesidad de intermediarios. Esto ayudará a garantizar que los productores locales puedan obtener un precio justo por sus productos y llegar a nuevos clientes, al mismo tiempo que proporciona a los compradores acceso a una mayor variedad de productos locales.

Nuestra aplicación ofrecerá una solución a los retos a los que se enfrentan los productores locales, creando una conexión directa entre ellos y los consumidores. La plataforma permitirá a los productores locales poner a la venta sus productos y conectar con compradores potenciales de su zona. De este modo, los productores tendrán acceso a un mayor número de clientes potenciales y podrán llegar a nuevos mercados.

La aplicación se diseñará pensando en la sencillez y la facilidad de uso, para que los productores puedan poner sus productos a la venta y los compradores puedan examinarlos y comprarlos.

En conjunto, la aplicación ofrecerá una solución integral que ayudará a los productores locales a vender sus productos directamente a los consumidores y a llegar a nuevos mercados. Nuestra plataforma facilitará que los productores se pongan en contacto con los compradores, obtengan un precio justo por sus productos y hagan crecer su negocio.

\section{Estudio de trabajo relacionado}\label{sec:estudio_trabajo_relacionado}

Al iniciar cualquier nuevo proyecto de software, es importante comprender el panorama del mercado e identificar los productos similares que ya existen. Vamos a realizar un análisis exhaustivo de otras plataformas que permiten a los productores locales vender sus productos directamente a los consumidores o, en su defecto, aplicaciones similares de compraventa. Este análisis nos ayudará a entender las principales características y funcionalidades de estos productos, así como a identificar cualquier vacío en el mercado que nuestra aplicación pueda cubrir. Al conocer lo que ya existe, podemos asegurarnos de que nuestra aplicación ofrezca un valor único a los usuarios y se diferencie de la competencia. Además, podemos utilizar esta información para diseñar y desarrollar nuestra aplicación, asegurándonos de que ofrece una experiencia de usuario atractiva y satisface las necesidades de los productores y compradores locales.

En una primera búsqueda encontramos que se nos ofrecen hasta decenas de aplicaciones móviles cuya finalidad es, en esencia, la misma que la de nuestro proyecto: que los productores locales puedan vender sus productos sin intermediarios. La mayoría de ellas surgen y operan únicamente en países en los que predominan en su mayoría las zonas rurales, como son India o Perú. Otras surgen en España, principalmente con la finalidad de exportar productos agroalimentarios al resto de Europa. Realizaremos un estudio a fondo de todas ellas repasando su trayectoria, las funcionalidades que ofrecen y los puntos fuertes y débiles de cada una de ellas.

\subsection{Naranjas Del Carmen}

Naranjas del Carmen es una plataforma web que lleva el nombre del mismo huerto donde se produce la cosecha en Valencia. El modelo de negocio principal es el del cultivo de naranjas ecológicas bajo demanda. Para poder pedir naranjas es necesario adoptar un naranjo de su huerta. Véase la figura \ref{fig:delcarmen-landing}

\figura{0.8}{img/naranjas-del-carmen/landing}{Página de inicio de Naranjas del Carmen}{fig:delcarmen-landing}{}

\subsubsection{Historia}

% TODO: El texto de abajo es parafraseado y resumido de https://www.naranjasdelcarmen.com/emprendedores. Como debería citarlo?
Los orígenes de la página web se remontan al año 2011, cuando los hermanos Gabriel (28) y Gonzalo (25) Úrculo deciden dejar sus trabajos para recuperar el huerto de su abuelo durante la crisis económica, que hasta entonces había estado abandonado. Tras conseguir ponerlo en marcha con la ayuda de un crédito y su familia, y tras una primera cosecha poco fructífera a causa de los intermediarios, los hermanos lanzan la plataforma web \url{https://www.naranjasdelcarmen.com/}. La finalidad de la web es la de vender los productos que cultivaban en su huerto de Valencia, Naranjas Del Carmen. En la primera temporada de la web envían naranjas a 150 hogares, principalmente familia, amigos y amigos de amigos.

En 2013 se colocan las primeras colmenas de abejas en el huerto. Las abejas polinizan las flores de azahar de sus naranjos y aportan auténtica miel de azahar directa del panal. Los hermanos comienzan también a vender miel, con 500 familias como clientes.

En 2014 comienzan con la producción de aceite de oliva virgen extra ecológico en Altura (Castellón), comienzan a renovar los naranjos arrancando árboles muertos y preparan los terrenos para nuevas plantaciones

En 2015, los hermanos Gabriel y Gonzalo deciden ampliar el negocio, dando empleo a 10 personas más y preparando una zona para una huerta. Cultivan fruta y verdura típicas de la huerta valenciana mediterránea. En este año también surge CrowdFarming, bajo la idea de que los naranjos se vayan plantando por encargo de las familias que les piden naranjas, y a cada naranjo se le cuelga un cartel con el nombre escogido por su dueño.

Para el año 2016 ya hay más de 1000 árboles con dueño. Cada uno de ellos se fotografía una vez al año para que los dueños puedan seguir su evolución. A partir de este momento, la plataforma de CrowdFarming comienza a crecer a buen ritmo: Se inicia la producción en abejas, con más de 200 personas adoptando una colmena en los campos de El Carmen.

En el año 2018 se unen nuevos agricultores a la ya establecida plataforma CrowdFarming, permitiendo a ciudadanos de todo Europa adoptar o plantar árboles de agricultores de cualquier parte del mundo.

Hasta el día de hoy, los fundadores de la plataforma Naranjas del Carmen han estado apostando por el crecimiento de la plataforma hermana, CrowdFarming, más accesible para los consumidores. Naranjas del Carmen sigue operando, subsidiada principalmente por el mantenimiento que pagan los clientes que han adoptado árboles y colmenas en la finca.

\subsubsection{Funcionalidades}

\begin{itemize}

	\item Adoptar de un árbol o colmena de la finca para recibir su producción anualmente (modelo "suscripción"). Véase la figura \ref{fig:delcarmen-landing}.

	\item Comprar cajas de productos de la huerta de El Carmen. Estos engloban productos mediterráneos, cítricos, caquis, tomates, naranjas, miel y un largo etcétera.

	\item El portal dispone de un blog de agricultura y apicultura con artículos periódicos. Esto ayuda a mantener un mejor posicionamiento en buscadores y a atraer potenciales clientes al portal. Véase la figura \ref{fig:delcarmen-blog}

\end{itemize}

\figura{0.8}{img/naranjas-del-carmen/blog}{Blog de Naranjas del Carmen}{fig:delcarmen-blog}{}

\subsubsection{Ventajas}

\begin{itemize}

	\item Productos ecológicos y de alta calidad: La plataforma ofrece naranjas, miel y aceite de oliva ecológicos y de alta calidad, cultivados en un huerto tradicional valenciano.

	\item Modelo innovador: La adopción de árboles de Naranjas Del Carmen permite a los consumidores participar en el cultivo y producción de sus propios productos, conociendo así su origen y la forma en que se cultivan e involucrándolos en su desarrollo.

	\item Fomento de la agricultura local: La plataforma apoya la agricultura local y tradicional para ofrecer productos frescos y de calidad.

	\item Conciencia ambiental: Al cultivar productos ecológicos y fomentar la agricultura local, la plataforma también contribuye a una conciencia ambiental y a un futuro más sostenible. Recientemente han reducido considerablemente el uso de plástico en sus productos.

	\item Sentido de comunidad: La plataforma fomenta una comunidad de agricultores y consumidores que comparten una pasión por la agricultura y los productos frescos y naturales.

\end{itemize}

\subsubsection{Inconvenientes}

\begin{itemize}

	\item Inversión inicial: Ese necesario adoptar un árbol o colmena para poder acceder a los productos, lo que puede hacer que la plataforma no esté al alcance de todos los consumidores. No obstante, es posible hacer pedidos de menor cantidad para probar la cosecha antes de adoptar un árbol o colmena.

	\item Requiere compromiso: La adopción de un árbol o colmena requiere un compromiso a largo plazo, lo que puede ser un obstáculo para algunos consumidores. Esencialmente, se convierte en un modelo de suscripción en el que hay que pagar anualmente para recibir la cosecha. Véase la figura \ref{fig:delcarmen-adoptar}.

\end{itemize}

\figura{0.8}{img/naranjas-del-carmen/adoptar-naranjo}{Página de adoptar un naranjo en Naranjas del Carmen}{fig:delcarmen-adoptar}{}

\subsection{CrowdFarming}

CrowdFarming es una plataforma web y móvil que permite comprar productos de temporada sin intermediarios, promoviendo una agricultura europea más humana y sostenible. La idea surge de los creadores de plataforma Naranjas del Carmen, con la diferencia de que CrowdFarming sigue un modelo de "marketplace"{} en el que no es necesario adoptar un árbol para poder comprar.

\subsubsection{Historia}

La historia de CrowdFarming se remonta al año 2015. Los hermanos Gabriel y Gonzalo Úrculo, tras el éxito de la plataforma analizada previamente, Naranjas del Carmen, deciden lanzar una plataforma web que permitiera a otros productores locales vender sus productos por internet. Con el tiempo ampliaron su negocio a la producción de miel de azahar y aceite de oliva virgen extra ecológico. La idea de CrowdFarming tuvo una respuesta positiva y rápidamente comenzaron a unirse nuevos agricultores a la plataforma, permitiendo a los ciudadanos de toda Europa poder adoptar árboles del campo para recibir su cosecha o comprar directamente cajas de productos de temporada a los agricultores. Con más de 1000 árboles con dueño y la creciente adopción de colmenas, la plataforma de CrowdFarming ha demostrado ser un modelo de negocio sostenible y comprometido con la agricultura ecológica y local.

Actualmente la plataforma da un espacio de venta a 272 productores de 8 países que venden sus productos directamente al consumidor. Entre todos suman más de 440.000 árboles adoptados y más de 3.680.000 cajas de productos frescos enviadas directamente del agricultor a los consumidores.

Desde la página principal de CrowdFarming se nos da la opción de adoptar un árbol o de comprar una caja de productos de temporada (véase la figura \ref{fig:crowd-impacto}). Esta segunda opción resulta interesante si no tenemos la intención de consumir los productos de forma recurrente o si queremos hacer un primer pedido de prueba para evaluar la calidad de la producción.

\figura{0.8}{img/crowdfarming/impacto}{Maneras de generar impacto mediante CrowdFarming}{fig:crowd-impacto}{}

% La cita de la web del informe está bien así? Falta información?
Tal y como reportan en su informe de impacto y transparencia de 2021 \cite{informe_crowdfarming}, los países que más compran productos de España en la plataforma son Alemania (52,68\%), Francia (8,44\%) y Austria (6,49\%). Con esto podemos identificar que la fuente de ingresos principal de CrowdFarming en la actualidad es la comisión por cada venta.

\subsubsection{Funcionalidades}

\begin{itemize}

	\item Adoptar un árbol o colmena de un productor local para recibir su producción anualmente (modelo "suscripción"). Véase la figura \ref{fig:crowd-impacto}.

	\item Comprar productos a agricultores independientes (modelo "marketplace"). Véase la figura \ref{fig:crowd-productos}.

	\item Planear compras recurrentes y establecer fechas para recibir las cosechas.

	\item Darse de alta como vendedor en la plataforma.

	\item Opción para empresas (se paga por empleado, quienes reciben la cosecha en su casa; o como oficina, para recibir paquetes de fruta de temporada en la misma). Véase la figura \ref{fig:crowd-empresas}

	\item La aplicación web también dispone de un blog de agricultura, recetas sostenibles y podcast. Con esto se fomenta el sentimiento de comunidad que promueve la aplicación y también ayuda a mantener un mejor posicionamiento en buscadores y a atraer potenciales clientes al portal. Véase la figura \ref{fig:crowd-blog}

\end{itemize}

\figura{0.8}{img/crowdfarming/empresas}{Planes para empresas de Crowdfarming}{fig:crowd-empresas}{}

\figura{0.8}{img/crowdfarming/blog}{Blog de Crowdfarming}{fig:crowd-blog}{}

\subsubsection{Ventajas}

\begin{itemize}

	\item Acceso a productos frescos y de calidad: Al comprar directamente de los agricultores, los consumidores tienen acceso a productos frescos y de calidad, cultivados de forma ecológica y sostenible.

	\item Apoyo a la agricultura local: Al comprar a través de CrowdFarming, los consumidores están apoyando a la agricultura local y reduciendo su huella de carbono, ya que los productos no tienen que ser transportados desde lejanas regiones.

	\item Oferta variada de productos: Dado que la plataforma no es centralizada, sino que hay agricultores en varias partes de Europa, es posible ofrecer diferentes tipos de alimentos de temporada (véase la figura \ref{fig:crowd-productos}).

	\item Menos intermediarios: El modelo de negocio de CrowdFarming favorece la ausencia de intermediarios físicos, lo que mejora la calidad de los alimentos cuando llegan al consumidor final y hace más rápida la llegada de los productos a los mismos.

	\item Producción de alimentos más frescos: Al reducir el número de intermediarios también se reduce el tiempo entre que se recolecta la producción y se consume, resultando en alimentos más frescos para el consumidor.

	\item Mayor transparencia en el proceso de producción: Al tener acceso directo a los agricultores, los consumidores pueden conocer mejor la forma en que se producen los productos y estar seguros de su calidad y origen.

	\item Presencia notable en internet: La web y las aplicaciones de Android e iOS tienen una estética cuidada y homogénea, lo que da confianza a los usuarios en la empresa. Además disponen de un equipo de marketing que se encarga de asegurar la presencia de CrowdFarming en redes sociales.

\end{itemize}

\subsubsection{Inconvenientes}

\begin{itemize}

	\item Presencia de intermediarios: Aunque la plataforma haga desaparecer los intermediarios físicos habituales que intervienen un supermercado, se convierte en sí misma en un intermediario que disminuye el rendimiento económico de la cosecha para el agricultor. El agricultor recibe el 50\% del precio de venta base mientras que el resto se divide en transporte (25\%), comisión del servicio (22\%) y comisión de los métodos de pago (3\%).

	\item Diferencias de calidad: Al ser una plataforma que une a muchos productores, la calidad de los productos puede variar de un productor a otro, lo que puede generar insatisfacción entre los consumidores.

	\item Costes de envío: El envío directo de los productos frescos desde el agricultor al consumidor puede resultar en costos de envío elevados, especialmente si el agricultor se encuentra muy lejos de una zona urbana.

\end{itemize}

\figura{0.8}{img/crowdfarming/productos}{Oferta de productos en CrowdFarming}{fig:crowd-productos}{}

\subsection{Farm To People}
% https://farmtopeople.com/about-us
Farm To People es una empresa que tiene como finalidad poner alimentos frescos al alcance de los habitantes de Nueva York. Ofrecen acceso a mercados de agricultores y granjeros para conseguir alimentos sostenibles. Su misión es la de construir un sistema alimentario justo, transparente, sostenible, ético, diverso y seguro. Disponen de una red de más de 150 granjas a menos de 500km de la ciudad de Nueva York y más de 800 productos distintos.

\subsubsection{Historia}

Farm To People nació en 2013 de una pasión compartida por padre e hijo por la comida saludable, fruto de pequeños productores que siguen métodos tradicionales.

Su fundador, David Robinov, ha sido un emprendedor en serie desde 1981. Con seis mercados minoristas naturales exitosos en el área de Nueva York también cofundó Organic Brands, LLC y desarrolló la línea de productos Mediterranean Organic. A través de Farm To People, David espera retribuir y apoyar a la próxima generación de pequeños agricultores. Es mentor y socio de su hijo, Michael, quien, después de un breve período en la Escuela de Artes Tisch de la Universidad de Nueva York, decidió seguir su pasión por la comida.

Con las grandes empresas tomando el control de la industria alimentaria, los fundadores David y Michael creen que es más importante que nunca hacer saber a la gente de dónde vienen los alimentos que consumen y quiénes son las personas que los preparan. Lo que buscan conseguir con Farm To People es que una plataforma que pueda ayudar a iniciar esta conversación y brindar negocios transparentes a los consumidores que deseen construir un mejor sistema alimentario.

Desde el comienzo de la pandemia ha tenido lugar un cambio importante en la plataforma, que ha dejado de ofrecer la posibilidad de hacer envíos nacionales para centrarse sólo en el reparto a domicilio en la ciudad de Nueva York.

Recientemente ha abierto en Brooklyn el "Farm to People Kitchen \& bar", que sirve platos de temporada elaborados únicamente con productos de agricultores y granjeros que colaboran con la plataforma.

\subsubsection{Funcionalidades}

\begin{itemize}

	\item Solicitar productos habituales mediante una suscripción semanal a los mismos, así como solicitar "Seasonal Produce Boxes"{} en diferentes tamaños con productos de temporada

	\item Suscribirse a una lista de correo para recibir noticias sobre productos de temporada, recetas e inspiraciones de cocina

	\item Donar comida a organizaciones que luchan contra el desperdicio de alimentos masivo y la desigualdad

	\item Comprar productos frescos de temporada, productos elaborados o bebidas a modo de compra única

	\item Hasta el día del envío se pueden añadir y eliminar elementos de la cesta. Cuando llegue este día, se procesarán y se cobrarán estos productos.

	\item La plataforma cuenta con un blog de recetas donde se plantean ideas y sugerencias en base a los ingredientes que se pueden adquirir en la web. Véase la figura \ref{fig:ftp-recetas}

\end{itemize}

\figura{0.8}{img/ftp/recetas}{Blog de recetas de Farm To People}{fig:ftp-recetas}{}

\subsubsection{Ventajas}

\begin{itemize}

	\item Envíos planificados: Es posible saber con antelación el día en el que se entregarán los alimentos

	\item Sistema de referidos: La plataforma ofrece \$20 a aquellos usuarios que inviten a otros amigos a la aplicación, dando a conocer el servicio.

	\item Comidas preparadas: En la web también se pueden comprar platos ya preparados para recalentar, elaborados todos ellos con productos ecológicos. Véase la figura \ref{fig:ftp-preparados}

\end{itemize}

\figura{0.8}{img/ftp/preparados}{Comidas preparadas en Farm To People}{fig:ftp-preparados}{}

\subsubsection{Inconvenientes}

\begin{itemize}

	\item Limitaciones geográficas: El servicio de reparto a domicilio sólo está disponible dentro de la ciudad de Nueva York por lo que actualmente no es posible comprar productos frescos desde otros estados
	
	\item Precios elevados: Al ser productos ecológicos y operar únicamente en la ciudad de Nueva York, el precio de los productos es bastante más caro de lo que se encontraría en un supermercado común, haciéndolo menos accesible al público general

\end{itemize}


\subsection{Bijak}

Bijak es una plataforma de comercio agrícola que ayuda a los comerciantes de la India a comprar y vender todo tipo de bienes relacionados con la agricultura. Cuenta con una comunidad de mas de 30.000 comerciantes clasificados en base a datos de sus transacciones realizadas, y cuenta también con los precios diarios de más de 2.000 mercados (mandis). La aplicación opera en 28 estados/territorios de unión de la India y cuenta con el apoyo del ministerio de ciencia y tecnología del gobierno.

\subsubsection{Historia}

Bijak se funda en Abril de 2019, y un mes después se lanza el MVP para una región y un único producto de comercialización. En septiembre-octubre del mismo año, la aplicación recaudó 2,5 millones de dólares en una ronda de financiación inicial y lanzó la segunda versión de la aplicación (véase la figura \ref{fig:bijak-landing}).

\figura{0.8}{img/bijak/landing}{Página principal de Bijak}{fig:bijak-landing}{}

En enero del 2020, Bijak ya tenía presencia en 16 estados y 200 regiones de la India, con un total de 47 productos disponibles. En abril del mismo año consiguen recaudar 12 millones de dólares para mejorar las capacidades tecnológicas de la aplicación.

Recientemente, en enero de 2022, la aplicación ha recaudado otros 19 millones de dólares, elevando la financiación total hasta el momento de 33,5 millones de dólares. Actualmente hay 110 millones de agricultores que dependen de comerciantes de productos básicos para subsistir, sector que supone el 14 por ciento del PIB del país.

\subsubsection{Funcionalidades}

%https://www.bijak.in/
\begin{itemize}

	\item Acceso a una red de compradores y proveedores agrícolas: Bijak conecta a los comerciantes agrícolas de la India con una red de compradores y proveedores. Los usuarios pueden buscar y encontrar a otros comerciantes agrícolas, lo que les permite ampliar su red y encontrar nuevas oportunidades de negocio

	\item Acceso a información de mercado: Bijak proporciona información diaria de más de 2.000 mercados (mandis) en toda la India, lo que permite a los usuarios obtener información sobre los precios y la demanda de los productos agrícolas en diferentes regiones.

	\item Acceso a detalles de las transacciones realizadas y almacenamiento de los documentos pertinentes en el dispositivo móvil: La aplicación permite a los usuarios almacenar y acceder a los detalles de las transacciones realizadas. Esto les permite llevar un registro de sus transacciones pasadas y recuperar información importante, como los precios de compra y venta, en cualquier momento; así como generar facturas. Véase la figura \ref{fig:bijak-facturas}

	\item Oferta de préstamos en tiempo real a los compradores para poder pagar al contado a los proveedores: Bijak ofrece préstamos en tiempo real a los compradores para que puedan pagar al contado a los proveedores y cerrar la transacción de manera inmediata. Esto acelera el proceso de comercio agrícola y ayuda a los usuarios a cerrar acuerdos más rápidamente.

	\item Recibir notificaciones relevantes y enviar recordatorios de pagos a otros usuarios: La aplicación envía notificaciones relevantes a los usuarios, como recordatorios de pagos pendientes, fechas de entrega y actualizaciones de precios. Los usuarios también pueden enviar recordatorios de pago a otros usuarios para asegurarse de que se cumplan los plazos.

	\item Hacer pagos rápidos y seguros a otros usuarios: Bijak permite a los usuarios hacer pagos rápidos y seguros a otros usuarios. Los usuarios pueden realizar pagos en línea o mediante la banca móvil, lo que facilita el proceso de comercio agrícola.

	\item Valorar al resto de usuarios de la comunidad mediante un sistema de reseñas: La aplicación cuenta con un sistema de reseñas que permite a los usuarios valorar a otros miembros de la comunidad. Esto ayuda a los usuarios a tomar decisiones informadas sobre con quién hacer negocios y también fomenta un comportamiento positivo dentro de la comunidad.

\end{itemize}

\figura{0.4}{img/bijak/facturas}{Generador de facturas electrónicas de Bijak}{fig:bijak-facturas}{}

\subsubsection{Ventajas}

\begin{itemize}

	\item Conectividad: La aplicación conecta a los productores agrícolas con otros compradores y vendedores en toda la India, lo que les permite expandir su alcance más allá de los mandis locales.

	\item Transparencia: La plataforma proporciona precios diarios de más de 2.000 mercados) en 28 de los estados y territorios de la unión de la India, lo que permite a los comerciantes tomar decisiones informadas sobre cuándo y cuánto comprar y vender su mercancía.

	\item Seguridad: La aplicación permite enviar y recibir pagos de forma segura y ofrece garantías al comprador. Tanto compradores como vendedores están sujetos a un sistema de reseñas por parte de otros usuarios que hayan tenido interacciones comerciales con ellos.

	\item Comunidad: Bijak cuenta con una comunidad de más de 30.000 comerciantes, brindando así a los usuarios una amplia red de contactos y oportunidades comerciales.

	\item Aplicación multiplataforma: Bijak dispone de aplicaciones tanto para Android como para iOS, por lo que es accesible a la inmensa mayoría de usuarios que dispongan de un teléfono inteligente.

	\item Apoyo del gobierno: Bijak cuenta con el apoyo del Ministerio de Ciencia y Tecnología del Gobierno de la India, lo que da a los usuarios más confianza en la aplicación y una sensación de seguridad.

\end{itemize}

\subsubsection{Inconvenientes}

\begin{itemize}

	\item Falta de regulación: Al ser una plataforma de comercio electrónico, puede haber una falta de regulación y protección contra prácticas comerciales engañosas o fraudulentas, más allá de la protección que ofrezca la propia aplicación.
	
	\item Desafíos logísticos: En el sector de la compraventa puede haber desafíos logísticos para la entrega de productos agrícolas desde diferentes partes de la India, lo que puede afectar la eficiencia y la rapidez del comercio.

	\item Limitaciones geográficas: La aplicación está fuertemente ligada al país en el cual opera, la India, haciendo más difícil una posible expansión al mercado internacional. Además de esto, solo contempla actualmente 28 estados y territorios de la unión, quedando 8 de ellos sin acceso a la aplicación.

	\item Necesidad de número de teléfono local: Actualmente no es posible explorar la aplicación sin antes registrarse con un número de teléfono de la India (véase la figura \ref{fig:bijak-login}).
	
\end{itemize}

\figura{0.3}{img/bijak/login}{Página de inicio de sesión de Bijak}{fig:bijak-login}{}

\subsection{Mandi Trades}
% https://www.livemint.com/Companies/yW1LDUCfWZDnnGYAPj1ddJ/Mandi-Trades-Sowing-the-seeds-of-genuine-profit.html

% https://www.livemint.com/Consumer/nQLEyDHTQvkVAodbfA6B9L/An-app-that-helps-farmers-cut-the-middleman-out.html
Mandi Trades es una aplicación diseñada para ayudar a los agricultores indios a vender sus productos directamente a los clientes. Los agricultores pueden registrarse en la aplicación e introducir datos sobre sus productos en venta, ubicación y precio, que luego se cargan en una base de datos escalable basada en la nube. La aplicación ofrece a los compradores una vista cartográfica de los productos disponibles, con datos sobre el producto y ordenados por proximidad geográfica al agricultor. En la actualidad ha cesado su operación.

\subsubsection{Historia}

La aplicación fue lanzada en 2014 por Farmmobi Technologies, con una inversión del equivalente a unos 45.000\geneuro{}  por parte de Edvin Varghese y Murthy Gurunathan, trabajadores de Oracle. El objetivo detrás de Mandi Trades era resolver el problema del monopolio de los intermediarios en los mandis estatales.

Aproximadamente un año después de su lanzamiento, la aplicación se lanzó en otros 5 idiomas (Hindi, Tamil, Telugu, Kannada y Malayalam) en su versión 2.0.

En 2016 la aplicación contaba con 30.000 usuarios en India. La inclusión de los smartphones en las zonas rurales ayudó a la empresa a ampliar su base de clientes y a lanzar más funcionalidades. De la misma forma, utilizaron redes sociales como Twitter y Facebook para dar a conocer la aplicación.

% https://tracxn.com/d/companies/mandi-trades/__MW24-PY9aB99fPoGlulole0IqhO8UUO4acp_SnuG9hk
En la actualidad, Mandi Trades ha cesado su operación por causas que no podemos identificar fácilmente. Su última interacción en redes sociales fue en abril de 2018, y el dominio web \url{https://manditrades.com} ya no se encuentra registrado por la empresa, que se encuentra inactiva. Esto puede deberse a que el uso de smartphones no estaba tan extendido en las zonas rurales como en la actualidad y a que la empresa no tenía ningún plan financiero establecido para generar beneficios.

\subsubsection{Funcionalidades}

\begin{itemize}

	\item Cuando un agricultor se registra en Mandi Trades, el sistema recoge información sobre sus productos y su ubicación y la almacena en una base de datos escalable basada en la nube.
	
	\item Para el comprador se ofrece una vista cartográfica de los productos disponibles con la información del producto, ordenada según su proximidad geográfica al agricultor.

	\item Hay opciones para ayudar a los agricultores a planificar mejor, como la visualización de los precios actuales de los productos básicos en el comercio, la demanda de los productos disponibles en la aplicación, el clima y los cambios estacionales.

\end{itemize}

\subsubsection{Ventajas}

\begin{itemize}

	\item Ofrece a los agricultores una plataforma para vender sus cosechas directamente a los compradores, lo cual puede hacerles obtener un mejor precio por sus cosechas al eliminar a los intermediarios.

	\item Proporciona información actualizada en tiempo real sobre los precios de los productos básicos y las tendencias del mercado.

	\item  Da comodidad a los agricultores, permitiéndoles vender sus cosechas desde cualquier lugar con conexión a Internet.

\end{itemize}

\subsubsection{Inconvenientes}

\begin{itemize}

	\item La aplicación requiere un smartphone y conexión a Internet para su uso, lo cual no está al alcance de todos los agricultores de la India, aún menos en el 2014 cuando se lanzó la aplicación.

	\item Los agricultores no están familiarizados con el uso de aplicaciones móviles de este tipo, lo que podría limitar su adopción. A esto se le suma una interfaz anticuada y con una experiencia de usuario poco cuidada. Véase la figura \ref{fig:mandi-interfaz}

	\item El transporte de las cosechas a los compradores puede plantear problemas logísticos.

	\item Podría haber problemas con los pagos y las disputas, ya que la aplicación no tiene presencia física para mediar en los conflictos.

\end{itemize}

\figura{0.8}{img/mandi-trades/interfaz}{Interfaz de Mandi Trades}{fig:mandi-interfaz}{}

\subsection{Kusikuy}
% https://www.gob.pe/institucion/minam/noticias/612964-aplicacion-digital-kusikuy-llevara-productos-de-nuestra-agrobiodiversidad-a-los-hogares-peruanos
Kusikuy es una aplicación móvil que permite la entrega a domicilio de una variedad de productos cosechados por más de 500 familias de agricultores conservacionistas de Cusco, Huancavelica, Puno y Apurímac. Detrás de la iniciativa se encuentra el ministro de Medio Ambiente de Perú, Modesto Montoya. El objetivo de esta iniciativa es hacer llegar a consumidores urbanos los productos nativos cultivados por productores rurales conservacionistas.

La finalidad de la aplicación es la de relacionar a productores y consumidores de productos nativos cultivados por agricultores que viven en zonas altoandinas, haciendo accesibles más de 70 especies nativas (véase la figura \ref{fig:kusikuy-interfaz}).

\figura{0.7}{img/kusikuy/interfaz}{Menú principal y productos a la venta en Kusikuy}{fig:kusikuy-interfaz}{}

\subsubsection{Historia}

La plataforma tiene una trayectoria corta hasta el momento: Se lanzó el 31 de mayo de 2022 con el apoyo del Ministerio del Ambiente, el Ministerio de Desarrollo Agrario y Riego y otras instituciones públicas del Perú. Ha experimentado un crecimiento acelerado, siendo posible descargarla tanto en dispositivos Android como en iOS y acumulando ya más de 50.000 descargas en la tienda de aplicaciones de Google.

\subsubsection{Funcionalidades}

\begin{itemize}

	\item Entrega a domicilio: Kusikuy permite a los usuarios comprar productos agrícolas nativos y sostenibles, los cuales son entregados directamente en sus hogares. Esto ayuda a fomentar el consumo de productos locales y la agricultura sostenible.

	\item Compra directa a los agricultores: Kusikuy permite a los agricultores vender directamente sus productos sin intermediarios, lo que mejora sus ingresos y fomenta la agricultura sostenible. La plataforma también ofrece una opción para que los agricultores puedan subir sus productos y administrar su inventario.

	\item Soporte para pagos digitales: Kusikuy ofrece una plataforma de pagos en línea para hacer las transacciones de manera segura y fácil.

	\item Información sobre los productos: La plataforma proporciona información detallada sobre los productos como su origen, variedad, características nutricionales y recetas de cocina. Esto ayuda a educar a los consumidores sobre la importancia de la agricultura sostenible y la diversidad de la gastronomía del país.

	\item Información sobre los agricultores: Kusikuy proporciona información sobre los agricultores que producen los alimentos, lo cual permite a los usuarios conocer las historias de los agricultores, su filosofía de cultivo, y otros detalles relevantes.

	\item Comentarios y calificaciones: Los usuarios pueden dar calificaciones y dejar comentarios sobre los productos y los agricultores que los producen. Esta funcionalidad ayuda a construir la confianza en la plataforma y a fomentar una comunidad de consumidores y productores.

\end{itemize}

\subsubsection{Ventajas}

\begin{itemize}

	\item Mayor mercado: Permite que los agricultores conservacionistas de zonas altoandinas de Perú accedan a un mercado más amplio, llegando a consumidores urbanos.

	\item Gran variedad: Da acceso a los consumidores de zonas urbanas a más de 70 especies nativas, que de otra manera podrían ser difíciles de encontrar en las ciudades. También es posible comprar productos ya elaborados (véase la figura \ref{fig:kusikuy-interfaz}).
	
	\item Agricultura sostenible: Ayuda a conservar de la biodiversidad en la región y apoya a los productores locales, que explotan sus tierras de forma más sostenible que los cultivos a gran escala.

	\item Menos intermediarios: Proporciona un canal de venta directo a los agricultores, evitando a los intermediarios comunes y resultando en una mayor parte del precio de venta que va a parar a los productores.

	\item Interfaz fácil de usar: La aplicación es intuitiva y fácil de usar, lo que la hace accesible para todos los usuarios. Los usuarios pueden navegar por las diferentes categorías, buscar productos específicos y agregarlos a su carrito de compras, haciéndola similar al resto de soluciones de e-commerce.

\end{itemize}

\subsubsection{Inconvenientes}

\begin{itemize}

	\item Plataforma nueva: La plataforma todavía no está muy extendida, lo que podría limitar la cantidad de consumidores y agricultores que participan en ella.

	\item Al alcance de algunos: Es posible que el alcance de la plataforma se limite a un público con un mayor poder adquisitivo y que esté dispuesto a pagar más por productos más exclusivos.
	
	\item Transporte poco fiable: Puede ser difícil garantizar la calidad y frescura de los productos entregados a domicilio si la paquetería no es confiable o puede tener retrasos excesivos por problemas de infraestructura en el país. La fecha de entrega aproximada es para dentro de 2 semanas a partir de la compra, por lo que se hace imposible la compraventa de productos perecederos.
	
	\item Situación política y económica: El éxito de la plataforma podría depender de factores externos, como el acceso a internet en las zonas rurales o la estabilidad política y económica del país, que actualmente es delicada y que ya ha causado algunos problemas en las operaciones de Kusikuy.

	\item Aplicación poco fiable: En mi caso personal de uso, al acceder a la aplicación tras varios días ha sido necesario borrar los datos de usuario y caché de la misma porque al abrir la aplicación se encontraba permanentemente en la pantalla de carga. Esto puede convertirse en un problema cuando usuarios menos experimentados usen la aplicación.

\end{itemize}
















     % !TEX root = ../proyect.tex

\chapter{Planificación}

% !TEX root = ../proyect.tex

\section{Planificación de tareas}\label{sec:planificacion-tareas}

En la tabla de horas dedicadas se puede ver el tiempo invertido en cada sección y subsección del proyecto, lo que permite hacer un seguimiento del progreso del trabajo y ajustar los plazos si es necesario. Para cada tarea se proporciona una descripción, definida por el nombre de la sección o subsección; el tiempo estimado, que en el caso ideal suma 300 horas; y el tiempo real, actualizado en cada entregable con los datos extraídos de Clockify.

\begin{table}[!ht]
    \centering
    \begin{tabular}{|l|l|l|}
    \hline
        \textbf{Descripción de la tarea} & \textbf{Tiempo estimado} & \textbf{Tiempo real} \\ \hline
        \textbf{Tiempo total} & \textbf{300 horas} & \textbf{241:10:20} \\ \hline
        \textbf{1.  Introducción} & \textbf{32 horas} & \textbf{31:58:17} \\ \hline
        1.1. Introducción al problema & 4 horas & 03:46:20 \\ \hline
        1.2. Descripción de la solución propuesta & 4 horas & 04:08:48 \\ \hline
        1.3. Estudio de trabajo relacionado & 20 horas & 21:01:40 \\ \hline
        1.4. Tabla comparativa del trabajo relacionado & 4 horas & 03:01:29 \\ \hline
        \textbf{2.  Planificación} & \textbf{1 hora} & \textbf{00:38:39} \\ \hline
        2.1. Planificación de tareas & 1 hora & 00:38:39 \\ \hline
        \textbf{3. Elicitación de requisitos} & \textbf{62 horas} & \textbf{55:44:02} \\ \hline
        3.1. Descripción de objetivos & 1 hora & 01:50:50 \\ \hline
        3.2. Actores del sistema & 1 hora & 01:26:00 \\ \hline
        3.3. Historias de usuario & 5 horas & 06:01:22 \\ \hline
        3.4. Requisitos de información & 2 horas & 05:17:06 \\ \hline
        3.5. Casos de uso & 10 horas & 15:51:59 \\ \hline
        3.6. Operaciones & 5 horas & 02:56:32 \\ \hline
        3.7. Matrices de trazabilidad & 10 horas & 04:59:49 \\ \hline
        3.8. Prototipos de interfaz & 25 horas & 16:27:50 \\ \hline
        3.9. Modelo de datos & 2 horas & 00:36:14 \\ \hline
        3.10. Requisitos no funcionales & 1 hora & 00:16:20 \\ \hline
        \textbf{4.  Diseño y arquitectura} & \textbf{25 horas} & \textbf{16:10:36} \\ \hline
        4.1. Diseño & 5 horas & 02:50:18 \\ \hline
        4.2. Arquitectura & 5 horas & 02:55:56 \\ \hline
        4.3. Stack tecnológico & 15 horas & 10:24:22 \\ \hline
        \textbf{5. Implementación} & \textbf{180 horas} & \textbf{136:38:46} \\ \hline
    \end{tabular}
    \caption{Tabla de planificación de tiempo por tarea}
    \label{tabla:planificacion}
\end{table}
     % !TEX root = ../../proyect.tex

\chapter{Elicitación de requisitos}\label{cap3}

% !TEX root = ../../proyect.tex

\section{Descripción de objetivos}\label{sec:descripcion-objetivos}

La aplicación tiene como objetivo principal proporcionar una plataforma eficiente y segura para que los productores puedan publicar y vender sus productos, y para que los usuarios puedan buscar y comprar productos de manera fácil y conveniente. Para lograr esto, se han establecido una serie de objetivos que se detallan a continuación:

\begin{enumerate}[label=OBJ-\protect\twodigits{\arabic*}:, align=left, leftmargin=*]
 
\item \textbf{Gestión de usuarios}: El sistema deberá permitir a los usuarios registrarse, iniciar sesión, editar su perfil y cerrar sesión en la aplicación. Además, el sistema debería ser capaz de gestionar los roles de los usuarios (usuario autenticado o productor) y restringir el acceso a ciertas funcionalidades según el rol del usuario.

\item \textbf{Gestión de productos}: El sistema deberá permitir a los productores publicar sus productos, editar la información de sus productos y eliminarlos. Los usuarios deben poder buscar y navegar entre los diferentes productos, así como ver detalles de los productos (precio, descripción, etc.).

\item \textbf{Gestión de pedidos}: El sistema deberá permitir a los usuarios comprar productos y realizar pagos, así como a los productores gestionar y aceptar/rechazar pedidos.

\item \textbf{Gestión de reseñas}: El sistema deberá permitir a los usuarios dejar reseñas y calificaciones para los productos que han adquirido, a modo de retroalimentación a los productores para mejorar la calidad de sus productos.

\item \textbf{Gestión de denuncias}: El sistema deberá permitir a los usuarios enviar reportes o denuncias a productos que no cumplan con las reglas de la aplicación, así como reportar problemas con sus pedidos, para asegurar la satisfacción de los clientes y poder solucionar problemas con los productores.

\item \textbf{Gestión de notificaciones}: El sistema deberá permitir a los usuarios autenticados recibir notificaciones relevantes relacionadas con su perfil, interacciones y eventos importantes en la aplicación. Los usuarios deben tener la capacidad de configurar sus preferencias de notificación y elegir los canales de comunicación (in-app, push, correo electrónico, SMS) a través de los cuales desean recibir las notificaciones.

\item \textbf{Gestión de conversaciones}: El sistema deberá permitir a los usuarios intercambiar mensajes con un sistema de chat y mensajería instantánea que facilite la comunicación entre los usuarios, permitiendo enviar y recibir mensajes de manera rápida y efectiva.

\end{enumerate}
% !TEX root = ../../proyect.tex

\section{Actores del sistema}\label{sec:actores-sistema}
En esta sección se presentan los diferentes actores que interactuarán con el sistema, es decir, aquellos usuarios o entidades que tendrán un papel en la ejecución de las funcionalidades de la aplicación. Los actores se definen en función de sus objetivos y responsabilidades en la aplicación. De esta manera, se establece un marco de referencia que permite entender las interacciones que se producen entre los diferentes elementos del sistema. Basándonos en los objetivos definidos para la aplicación, podemos identificar los siguientes actores:

\begin{enumerate}[label=ACT-\protect\twodigits{\arabic*}:, align=left, leftmargin=*]

    \item \textbf{Administrador}: Este actor tiene el papel de gestionar los recursos y realizar tareas de administración en la aplicación. Tiene acceso al panel de administrador y puede listar, modificar y eliminar productos, usuarios y pedidos. Puede acceder a la información de contacto de los usuarios, tal como su correo electrónico o su número de teléfono, con el fin de poder contactar con los mismos en caso de detectar por su parte un uso sospechoso de la aplicación.

    \item \textbf{Usuario no autenticado}: Este actor representa a las personas que utilizan la aplicación sin haber iniciado sesión. Pueden registrarse, iniciar sesión o navegar la aplicación con normalidad a modo de \textquote{solo lectura}. Esto significa que podrán ver perfiles de productores o publicaciones, pero no podrán iniciar conversaciones con productores o marcar publicaciones como favoritas
    
    \item \textbf{Usuario autenticado}: Este actor representa a los usuarios de la aplicación que tienen su sesión abierta. Pueden editar su perfil y cerrar sesión, además de iniciar conversaciones de chat con productores, emitir denuncias o dejar reseñas y calificaciones cuando un productor informe de que le ha vendido un producto. Además de esto, \textbf{pueden llevar a cabo las mismas operaciones de lectura que los usuarios no registrados}, como consultar publicaciones o perfiles de productores. Los usuarios autenticados pueden tener además el rol de productor.
    
    \item \textbf{Productor}: Este actor dispone de un perfil público y tiene la capacidad de publicar, editar y eliminar sus productos en la aplicación. Puede marcar sus publicaciones como vendidas y recibir reseñas de los consumidores por cada una de estas ventas. \textbf{Un usuario autenticado podrá convertirse en productor cuando complete su perfil}

\end{enumerate}
% !TEX root = ../../proyect.tex

\section{Historias de usuario}\label{sec:historias-usuario}

En esta sección, se presentarán las historias de usuario de la aplicación que se va a desarrollar. Las historias de usuario son descripciones detalladas de funcionalidades que deben ser implementadas en la aplicación desde el punto de vista del usuario. Cada historia de usuario representa una necesidad específica del usuario que la aplicación debe satisfacer. Estas historias se utilizarán como base para el desarrollo de la aplicación y guiarán todo el proceso de diseño y desarrollo.

\begin{enumerate}[HU-1:]

\subsection{Administradores}

\item \textbf{Gestionar publicaciones}: Como administrador, quiero poder eliminar publicaciones inapropiadas o spam en la plataforma para mantener la calidad de la experiencia de usuario.

\item \textbf{Gestionar categorías}: Como administrador, quiero poder agregar nuevas categorías de productos para que los productores puedan publicar productos más específicos.

\item \textbf{Gestionar pedidos}: Como administrador, quiero ver una lista de todos los pedidos realizados en la plataforma, para poder asegurarme de que se están cumpliendo correctamente.

\item \textbf{Gestionar usuarios}: Como administrador, quiero poder ver una lista de todos los usuarios registrados en la plataforma, para poder verificar su identidad y asegurarme de que cumplen con los requisitos para ser usuarios de la plataforma.

\item \textbf{Gestionar quejas y reclamaciones}: Como administrador, quiero poder gestionar las quejas y reclamaciones de los usuarios, para poder resolver los problemas de manera efectiva y garantizar la satisfacción del usuario.

\item \textbf{Crear promociones}: Como administrador, quiero poder crear promociones para productos específicos en la plataforma, para fomentar la venta de ciertos productos y mejorar la rentabilidad de la plataforma.

\item \textbf{Ver estadísticas y análisis de la plataforma}: Como administrador, quiero poder ver estadísticas y análisis de la plataforma, para poder tomar decisiones estratégicas y mejorar la rentabilidad de la plataforma.

\subsection{Productores}

\item \textbf{Crear publicaciones}: Como productor, quiero poder publicar mis productos en venta para que los consumidores locales puedan ver lo que tengo disponible y ponerse en contacto conmigo para realizar una compra en persona.

\item \textbf{Gestionar ventas}: Como productor, quiero poder gestionar las ventas realizadas y ver el historial de ventas para llevar un control de mis ingresos.

\item \textbf{Gestionar disponibilidad}: Como productor, quiero poder actualizar la disponibilidad de mis productos en tiempo real para evitar que los consumidores se interesen por productos que ya no tengo disponibles.

\item \textbf{Gestionar reservas}: Como productor, quiero poder marcar mis publicaciones como reservadas o vendidas para evitar confusiones o malentendidos con los consumidores.

\item \textbf{Publicar una cosecha en venta}: Como productor, quiero poder publicar en la plataforma la cosecha que tengo disponible para su venta, incluyendo información como el tipo de producto, cantidad, precio y fecha de recolección, para que los consumidores interesados puedan contactarme para realizar el intercambio en persona.

\item \textbf{Actualizar la información de una cosecha en venta}: Como productor, quiero poder editar la información de una cosecha que ya he publicado en la plataforma, para corregir errores o actualizar la información de la disponibilidad del producto.

\item \textbf{Eliminar una cosecha en venta}: Como productor, quiero poder eliminar una cosecha que ya no está disponible para la venta, para evitar que los consumidores intenten contactarme para realizar una compra.

\item \textbf{Recibir notificaciones de nuevos pedidos}: Como productor, quiero recibir una notificación en la plataforma o por correo electrónico cuando un consumidor realiza un nuevo pedido de alguno de mis productos, para que pueda contactarlo y coordinar la entrega.

\item \textbf{Visualizar el historial de pedidos}: Como productor, quiero poder ver el historial de pedidos realizados por los consumidores a través de la plataforma, para poder llevar un registro y mejorar mi planificación de la producción futura.

\subsection{Consumidores}

\item \textbf{Ver lista de productos en venta}: Como consumidor, quiero poder ver una lista de productos destacados venta para poder decidir qué productos comprar.

\item \textbf{Buscar productos}: Como consumidor, quiero poder buscar mediante un buscador productos locales en mi zona para poder apoyar a los productores locales y comprar productos frescos y de calidad.

\item \textbf{Filtrar lista de productos por categoría}: Como consumidor, quiero poder filtrar la lista de productos en venta por categoría (por ejemplo, frutas, verduras, carne, lácteos...) para poder encontrar productos específicos.

\item \textbf{Ver detalles de un producto}: Como consumidor, quiero poder ver los detalles de un producto en particular (descripción, precio, ubicación del productor) para poder tomar una decisión informada sobre si quiero comprar ese producto o no.

\item \textbf{Agregar producto al carrito de compras}: Como consumidor, quiero poder agregar productos al carrito de compras para poder comprar varios productos a la vez.

\item \textbf{Editar cantidad de producto en el carrito de compras}: Como consumidor, quiero poder editar la cantidad de un producto en el carrito de compras para poder ajustar la cantidad de productos que voy a comprar.

\item \textbf{Eliminar producto del carrito de compras}: Como consumidor, quiero poder eliminar productos del carrito de compras para poder comprar solo los productos que realmente quiero.

\item \textbf{Realizar el pago de los productos en el carrito de compras}: Como consumidor, quiero poder realizar el pago de los productos en el carrito de compras para poder finalizar la compra y recibir los productos.

\item \textbf{Ver historial de compras}: Como consumidor, quiero poder ver mi historial de compras para poder recordar qué productos he comprado anteriormente.

\item \textbf{Contactar con productores}: Como consumidor, quiero poder contactar directamente con el agricultor para acordar un punto de encuentro y realizar la compra en persona.

\item \textbf{Ubicación de productores}: Como consumidor, quiero poder ver la ubicación de los productores en un mapa para poder encontrar el punto de encuentro más cercano.

\item \textbf{Gestionar reseñas}: Como consumidor, quiero poder dejar una reseña o comentario en la publicación de un productor para ayudar a otros consumidores a tomar una decisión informada.

\end{enumerate}
% !TEX root = ../../proyect.tex

\section{Requisitos de información}\label{sec:requisitos-informacion}

En esta sección se definirán los atributos que serán utilizados por el sistema, sirviendo como punto de partida para el desarrollo de la base de datos y la interfaz de usuario. En el caso de la aplicación, los principales objetos que estarán presentes son los siguientes:

\begin{enumerate}[label=IRQ-\protect\twodigits{\arabic*}:, align=left, leftmargin=*]	

    \item \textbf{Administrador}: El sistema deberá almacenar la información correspondiente a los administradores del sistema. En concreto:
    \begin{itemize}
        \item \texttt{integer}. Identificador de administrador
        \item \texttt{string}. Correo electrónico
        \item \texttt{string}. Contraseña
    \end{itemize}
    
    \item \textbf{Usuario}: El sistema deberá almacenar la información correspondiente a los usuarios del sistema. En concreto:
    \begin{itemize}
		\item \texttt{integer}. Identificador de usuario
        \item \texttt{date}. Fecha de creación
        \item \texttt{date}. Fecha de última modificación
        \item \texttt{string}. Correo electrónico
		\item \texttt{string}. Contraseña
        \item \texttt{string}. Nombre
        \item \texttt{string}. Apellido(s)
		\item \texttt{string}. Número de teléfono
		\item \texttt{point}. Ubicación
		\item \texttt{image}. Foto de perfil
    \end{itemize}

    \item \textbf{Productor}: El sistema deberá almacenar la información correspondiente a los productores del sistema. En concreto:
    \begin{itemize}
		\item \texttt{integer}. Identificador de productor
        \item \texttt{string}. Biografía
    \end{itemize}
    
    \item \textbf{Publicación}: El sistema deberá almacenar la información correspondiente a cada producto publicado por un productor. En concreto:
    \begin{itemize}
        \item \texttt{integer}. Identificador de publicación
        \item \texttt{date}. Fecha de creación
        \item \texttt{date}. Fecha de última modificación
        \item \texttt{string}. Título
        \item \texttt{string}. Descripción
        \item \texttt{string}. Unidad (al peso o unitario)
        \item \texttt{integer}. Precio por unidad (en céntimos)
        \item \texttt{integer}. Stock disponible
        \item \texttt{list(image)}. Lista de imágenes
        \item \texttt{list(string)}. Lista de alérgenos
        \item \texttt{list(string)}. Lista de características
        \item \texttt{boolean}. Indicador de activo
    \end{itemize}

    \item \textbf{Pedido}: El sistema deberá almacenar la información correspondiente a los pedidos realizados por los consumidores. En concreto:
    \begin{itemize}
        \item \texttt{integer}. Identificador de pedido
        \item \texttt{date}. Fecha de creación
        \item \texttt{Publicación}. Publicación sobre la que se realiza el pedido
        \item \texttt{Usuario}. Usuario que realiza el pedido
        \item \texttt{float}. Cantidad de producto adquirida
        \item \texttt{float}. Precio total de la venta
    \end{itemize}

    \item \textbf{Reseña}: El sistema deberá almacenar la información correspondiente a las calificaciones realizadas a los productores por parte de los compradores de sus publicaciones. En concreto:
    \begin{itemize}
        \item \texttt{integer}. Identificador de reseña
        \item \texttt{date}. Fecha de creación
        \item \texttt{date}. Fecha de última modificación
        \item \texttt{integer}. Puntuación (1-5 estrellas)
        \item \texttt{string}. Comentario
        \item \texttt{Usuario}. Usuario que realiza la reseña
        \item \texttt{Publicación}. Publicación sobre la que se realiza la reseña
    \end{itemize}

    \item \textbf{Denuncia}: El sistema deberá almacenar la información correspondiente a las denuncias o reportes de problemas emitidos por los usuarios, referentes bien a un productor o bien a un pedido. En concreto:
    \begin{itemize}
        \item \texttt{integer}. Identificador de denuncia
        \item \texttt{date}. Fecha de creación
        \item \texttt{date}. Fecha de última modificación
        \item \texttt{Usuario}. Usuario emisor de la denuncia
        \item \texttt{Usuario}. Usuario receptor de la denuncia
        \item \texttt{string}. Descripción del problema
        \item \texttt{boolean}. Indicador de denuncia ya resuelta
        \item \texttt{Pedido}. Referencia al pedido denunciado (opcional)
    \end{itemize}

    \item \textbf{Conversación}: El sistema deberá almacenar la información correspondiente a cada conversación que mantengan un productor y un consumidor en lo referente a un producto. En concreto:
    \begin{itemize}
        \item \texttt{integer}. Identificador de conversación
        \item \texttt{date}. Fecha de creación
        \item \texttt{date}. Fecha de última modificación
        \item \texttt{Usuario}. Comprador asociado a la conversación
        \item \texttt{Publicación}. Publicación sobre la cual trata la conversación
    \end{itemize}

    \item \textbf{Mensaje}: El sistema deberá almacenar la información correspondiente a cada mensaje enviado por un productor o un consumidor en una conversación. En concreto:
    \begin{itemize}
        \item \texttt{integer}. Identificador de mensaje
        \item \texttt{date}. Fecha de creación
        \item \texttt{date}. Fecha de última modificación
        \item \texttt{string}. Tipo de mensaje
        \item \texttt{string}. Texto
        \item \texttt{string}. Datos (Campo JSON flexible para mensajes más sofisticados que texto plano)
        \item \texttt{Usuario}. Remitente
        \item \texttt{Usuario}. Destinatario
        \item \texttt{boolean}. Indicador de leído por el destinatario
    \end{itemize}

    
    \item \textbf{Notificación}: El sistema deberá almacenar la información correspondiente a cada notificación emitida a un usuario. En concreto:
    \begin{itemize}
        \item \texttt{integer}. Identificador de notificación
        \item \texttt{date}. Fecha de creación
        \item \texttt{date}. Fecha de última modificación
        \item \texttt{string}. Tipo de notificación
        \item \texttt{string}. Campo flexible de datos concretos de la notificación
        \item \texttt{Usuario}. Destinatario
        \item \texttt{boolean}. Indicador de leído por el destinatario
    \end{itemize}

\end{enumerate}

\figura{0.5}{img/diagramas/irq}{Diagrama de las relaciones entre los requisitos de información}{fig:diagrama-irq}{}

\todo{Esta es la primera versión, exportar el nuevo de PlantUML}
% !TEX root = ../../proyect.tex

\section{Casos de uso}\label{sec:casos-uso}
En esta sección, se presentan los casos de uso para la plataforma en línea de venta de productos. Estos casos de uso se han diseñado para permitir a los usuarios comprar y vender productos de forma eficiente y efectiva. Los casos de uso proporcionan una descripción detallada de las diferentes funcionalidades de la plataforma, y se han organizado en función del tipo de usuario que interactúa con el sistema.

Estos casos de uso describen los diferentes escenarios de interacción entre los usuarios (actores) y el sistema. Cada caso de uso describe una acción específica que el usuario puede realizar y cómo el sistema responde a ella, incluyendo los pasos que el usuario debe seguir para completar la acción, y se detallan posibles excepciones y flujos alternativos que el sistema puede tomar.

\begin{enumerate}[CU-1:]
\item{Registro de usuario}
\begin{itemize}
	\item \textbf{Actor:} Usuario
	\item \textbf{Descripción:} Permite al usuario crear una cuenta en la aplicación.
	\item \textbf{Flujo básico de eventos:}
	\begin{enumerate}[1:]
		\item El usuario accede a la página de registro.
		\item El sistema muestra un formulario de registro con los campos obligatorios: nombre, correo electrónico y contraseña.
		\item El usuario rellena el formulario con sus datos personales y confirma su registro.
		\item El sistema verifica que los datos sean válidos, crea una cuenta para el usuario y lo redirige a la página de inicio de sesión
	\end{enumerate}
	\item \textbf{Flujos alternativos:}
		\begin{itemize}
		\item[4a.] Si el correo electrónico ya está registrado, el sistema muestra un mensaje de error y pide al usuario que proporcione otro correo electrónico.
		\end{itemize}
\end{itemize}

\item{Inicio de sesión}
\begin{itemize}
	\item \textbf{Actor:} Usuario registrado
	\item \textbf{Descripción:} Permite al usuario iniciar sesión en su cuenta de la aplicación.
	\item \textbf{Flujo básico de eventos:}
	\begin{enumerate}[1:]
		\item El usuario accede a la página de inicio de sesión.
		\item El sistema muestra un formulario para introducir el correo electrónico y la contraseña del usuario.
		\item El usuario introduce su correo electrónico y su contraseña.
		\item El sistema comprueba que los datos son correctos, concede el acceso a la cuenta del usuario y lo redirige a su página de inicio.
		\end{enumerate}
	\item \textbf{Flujos alternativos:}
		\begin{itemize}
		\item[4a.] Si los datos son incorrectos, el sistema muestra un mensaje de error y solicita que se vuelvan a introducir.
		\end{itemize}
	\item \textbf{Excepciones:}
		\begin{itemize}
		\item[4a.] Si el sistema detecta algún problema técnico, muestra un mensaje de error y pide al usuario que lo intente más tarde.
		\end{itemize}
\end{itemize}

\item{Edición de perfil de usuario}
\begin{itemize}
	\item \textbf{Actor:} Usuario
	\item \textbf{Descripción:} Permite al usuario editar su información personal y de contacto.
	\item \textbf{Flujo básico de eventos:}
	\begin{enumerate}[1:]
		\item El usuario accede a la sección de \textquote{Editar perfil} en su cuenta.
		\item El sistema muestra la información personal y de contacto del usuario.
		\item El usuario edita los campos que desee modificar y confirma los cambios.
		\item El sistema actualiza la información del perfil del usuario.
		\end{enumerate}
	\item \textbf{Flujos alternativos:}
		\begin{itemize}
		\item[3a.] Si el usuario desea cancelar la edición de su perfil, puede seleccionar la opción de \textquote{Cancelar} y el sistema no realizará cambios en la información del perfil.
		\item[4a.] Si el usuario no completa algún campo obligatorio, el sistema mostrará un mensaje de error y no permitirá la confirmación de los cambios hasta que el campo sea completado correctamente.
		\end{itemize}
\end{itemize}

\item{Gestión de usuarios}
\begin{itemize}
	\item \textbf{Actor:} Administrador
	\item \textbf{Descripción:} Permite al administrador gestionar la información de los usuarios registrados en la plataforma.
	\item \textbf{Flujo básico de eventos:}
	\begin{enumerate}[1:]
		\item El administrador accede a la sección de gestión de usuarios.
		\item El sistema muestra una lista de todos los usuarios registrados en la plataforma.
		\item El administrador selecciona un usuario de la lista para ver su información.
		\item El sistema muestra la información del usuario seleccionado, incluyendo su nombre, correo electrónico, fecha de registro y historial de compras en el caso de un consumidor (o ventas y productos publicados en el caso de un productor).
		\item El administrador tiene la opción de editar la información del usuario o de eliminarlo de la plataforma.
		\item El sistema actualiza la lista de usuarios después de realizar cualquier cambio.
		\end{enumerate}
	\item \textbf{Flujos alternativos:}
	\begin{itemize}
		\item [5a.] Si se elige bloquear o eliminar un usuario, el sistema solicita confirmación antes de llevar a cabo la acción.
		\end{itemize}
\end{itemize}

\item{Gestión de productos}
\begin{itemize}
	\item \textbf{Actor:} Administrador
	\item \textbf{Descripción:} Permite al administrador gestionar los productos de la plataforma.
	\item \textbf{Flujo básico de eventos:}
	\begin{enumerate}[1:]
		\item El administrador accede al panel de administración de productos.
		\item El sistema muestra la lista de productos disponibles en la plataforma.
		\item El administrador puede modificar información de los productos existentes (nombre, descripción, precio, categoría, etc.) o eliminarlos de la plataforma.
		\item El administrador confirma la acción correspondiente.
		\item El sistema registra la acción y muestra un mensaje de confirmación.
		\end{enumerate}
	\item \textbf{Flujos alternativos:}
		\begin{itemize}
		\item [2a.] Si no hay productos registrados en la plataforma, el sistema muestra un mensaje indicando que no hay productos disponibles.
		\item [3a.] Si el administrador intenta realizar una acción no permitida (por ejemplo, eliminar un producto que tiene órdenes pendientes o que no le pertenece), el sistema muestra un mensaje de error indicando que la acción no es posible.
		\end{itemize}
\end{itemize}

\item{Gestión de pedidos}
\begin{itemize}
	\item \textbf{Actor:} Administrador
	\item \textbf{Descripción:} Permite al administrador gestionar los pedidos recibidos en la plataforma.
	\item \textbf{Flujo básico de eventos:}
	\begin{enumerate}[1:]
		\item El administrador accede a la sección de pedidos en su panel de control.
		\item El sistema muestra una lista con los pedidos recibidos, indicando el número de pedido, la fecha, el comprador y el estado del pedido.
		\item El administrador selecciona el pedido que desea gestionar.
		\item El sistema muestra la información detallada del pedido, incluyendo el consumidor y el productor involucrados, los productos comprados y el método de pago utilizado.
		\item El administrador modifica los detalles del pedido según corresponda.
		\item El sistema actualiza el estado del pedido en la base de datos y notifica al comprador del cambio de estado.
		\item El administrador puede añadir notas internas al pedido si lo considera necesario.
		\item El sistema registra las notas internas en la base de datos y las presenta en la sección correspondiente del pedido.
		\end{enumerate}
	\item \textbf{Flujos alternativos:}
		\begin{itemize}
		\item [5a.] Si el administrador detecta algún problema con el pago o la dirección de envío, puede contactar al comprador para solicitar la corrección de los datos.
		\item [6a.] Si el administrador cancela el pedido, el sistema emite un reembolso al comprador si corresponde y actualiza el inventario de los productos en la base de datos.
		\end{itemize}
\end{itemize}

\item{Búsqueda de productos}
\begin{itemize}
	\item \textbf{Actor:} Consumidor
	\item \textbf{Descripción:} Permite al consumidor buscar productos en la aplicación.
	\item \textbf{Flujo básico de eventos:}
	\begin{enumerate}[1:]
		\item El consumidor accede a la página de búsqueda de productos.
		\item El sistema presenta una barra de búsqueda.
		\item El consumidor introduce el término de búsqueda.
		\item El sistema muestra una lista de palabras clave que coinciden con el término de búsqueda
		\item El consumidor hace clic en el botón \textquote{buscar}
		\item El sistema muestra una lista de productos incluyendo su primera imagen, título, precio y distancia
		\item El consumidor hace clic en el producto para acceder a su página de información.
		\item El sistema muestra la información del producto, incluyendo sus imágenes, título, descripción, categoría, precio, disponibilidad y ubicación.
		\end{enumerate}
	\item \textbf{Flujos alternativos:}
	\begin{enumerate}
		\item[6a.] Si no se encuentra ningún producto que coincida con el término de búsqueda, el sistema muestra un mensaje indicando que no se han encontrado resultados.
		\end{enumerate}
	\item \textbf{Excepciones:}
	\begin{enumerate}
		\item[6a.] Si el término de búsqueda introducido por el consumidor contiene caracteres no alfanuméricos, el sistema muestra un mensaje de error indicando que el término de búsqueda es incorrecto.
		\end{enumerate}
\end{itemize}

\item{Publicación de productos}
\begin{itemize}
	\item \textbf{Actor:} Productor
	\item \textbf{Descripción:} Permite al productor publicar un nuevo producto en la plataforma.
	\item \textbf{Flujo básico de eventos:}
	\begin{enumerate}[1:]
		\item El productor accede al apartado de \textquote{Publicar producto} en su perfil.
		\item El sistema muestra un formulario con los campos para introducir la información del producto: imágenes, título, descripción, categoría, precio, disponibilidad y ubicación
		\item El productor completa los campos y envía el formulario.
		\item El sistema valida la información y registra el nuevo producto en la base de datos y muestra una confirmación de que el producto ha sido publicado con éxito.
		\end{enumerate}
	\item \textbf{Flujos alternativos:}
		\begin{itemize}
		\item[2a.] El productor decide no publicar el producto: Si el productor decide no publicar el producto, puede cancelar el proceso y volver al menú principal.
		\item[3a.] El formulario contiene errores: Si el formulario contiene errores o información incompleta, el sistema muestra una notificación de error y pide al productor que corrija los campos necesarios.
		\end{itemize}
\end{itemize}

\item{Compra de productos}
\begin{itemize}
	\item \textbf{Actor:} Consumidor
	\item \textbf{Descripción:} Permite al consumidor adquirir un producto específico.
	\item \textbf{Precondiciones:} El consumidor ha iniciado sesión y ha buscado y seleccionado el producto que desea adquirir.
	\item \textbf{Flujo básico de eventos:}
	\begin{enumerate}[1:]
		\item El consumidor accede a la página del producto que desea adquirir.
		\item El sistema muestra la información del producto, incluyendo su precio y la opción de adquirirlo.
		\item El consumidor selecciona la opción de adquirir el producto.
		\item El sistema muestra al consumidor la confirmación de la compra y el precio total.
		\item El consumidor confirma la compra.
		\item El sistema registra la compra, la hace llegar al productor correspondiente y presenta al consumidor la información de contacto del productor.
		\end{enumerate}
	\item \textbf{Flujos alternativos:}
	\begin{enumerate}
		\item [2a.] Si el sistema detecta que el producto ya ha sido vendido, se mostrará un mensaje de que el producto ya no está disponible.
		\item [4a.] Si el consumidor cambia de opinión o detecta un error, puede cancelar la compra en lugar de confirmarla.
		\item [7a.] Si el consumidor no puede contactar con el productor en un plazo después de la compra, se le dará la opción de cancelar la compra y recibir un reembolso en el caso de haber pagado mediante la aplicación
		\end{enumerate}
	\item \textbf{Excepciones:}
	\begin{enumerate}
		\item[5a.] Si el consumidor no ha iniciado sesión, se le solicitará que inicie sesión antes de poder adquirir el producto.
		\item[6a.] Si el sistema detecta que el consumidor ha proporcionado información de pago incorrecta o no válida, se le solicitará que proporcione información de pago correcta antes de poder confirmar la compra.
		\end{enumerate}
\end{itemize}

\item{Cancelación de pedido}
\begin{itemize}
\item \textbf{Actor:} Consumidor
\item \textbf{Descripción:} Permite al consumidor cancelar un pedido antes de su confirmación por parte del productor
\item \textbf{Flujo básico de eventos:}
	\begin{enumerate}[1:]
\item El consumidor accede a su historial de pedidos.
\item El sistema muestra una lista de los pedidos realizados por el consumidor.
\item El consumidor selecciona el pedido que desea cancelar.
\item El sistema muestra la información del pedido, incluyendo la opción de cancelarlo.
\item El consumidor confirma la cancelación del pedido.
\item El sistema cancela el pedido y emite un reembolso al consumidor si corresponde.
\end{enumerate}
\item \textbf{Flujos alternativos:}
\begin{enumerate}
\item [4a.] Si el pedido ya ha sido confirmado por el productor, el sistema notifica al consumidor que no es posible cancelar el pedido.
\item [5a.] Si se ha realizado un pago con tarjeta de crédito, el reembolso puede tardar varios días hábiles en aparecer en la cuenta del consumidor.
\end{enumerate}
\end{itemize}

\item{Consulta de pedidos recibidos}
\begin{itemize}
	\item \textbf{Actor:} Productor
	\item \textbf{Descripción:} Permite al productor ver los pedidos recibidos por sus productos.
	\item \textbf{Flujo básico de eventos:}
	\begin{enumerate}[1:]
		\item El productor accede a la sección de \textquote{Pedidos recibidos} en su perfil de usuario.
		\item El sistema muestra una lista de los pedidos recibidos, ordenados por fecha de solicitud.
		\item El productor selecciona un pedido de la lista para ver más detalles.
		\item El sistema muestra la información del pedido, incluyendo el nombre y la dirección del comprador, los productos solicitados, el precio total y el estado actual del pedido.
		\end{enumerate}
	\item \textbf{Flujos alternativos:}
		\begin{itemize}
		\item [2a.] Si el productor no tiene ningún pedido recibido, el sistema muestra un mensaje indicando que no hay pedidos disponibles.
		\item [3a.] Si el productor selecciona un pedido que ha sido cancelado o finalizado, el sistema muestra la información del pedido pero indica que ya no puede ser modificado.
		\end{itemize}
\end{itemize}

\item{Rechazo de pedido}
\begin{itemize}
\item \textbf{Actor:} Productor
\item \textbf{Descripción:} Permite al productor rechazar un pedido recibido del consumidor.
\item \textbf{Precondición:} El productor ha recibido un pedido del consumidor.
\item \textbf{Flujo básico de eventos:}
\begin{enumerate}[1:]
\item El productor accede a la lista de pedidos recibidos.
\item El sistema muestra la lista de pedidos recibidos.
\item El productor selecciona el pedido que desea rechazar.
\item El sistema muestra la información del pedido, incluyendo la opción de rechazarlo.
\item El productor selecciona la opción de rechazar el pedido.
\item El sistema muestra un mensaje de confirmación para rechazar el pedido y da la posibilidad al productor de añadir una nota para el consumidor.
\item El productor confirma la acción.
\item El sistema registra la cancelación del pedido y notifica al comprador.
\end{enumerate}
\item \textbf{Flujos alternativos:}
\begin{itemize}
\item[5a.] Si el pedido ya ha sido entregado al comprador, el productor no podrá rechazarlo.
\item[7a.] Si el productor decide no rechazar el pedido, éste sigue su curso.
\end{itemize}
\end{itemize}
\end{enumerate}

% !TEX root = ../../proyect.tex

\section{Operaciones}\label{sec:operaciones}

En esta sección se detallarán las operaciones que conformarán los casos de uso de de la aplicación. Las operaciones son un conjunto de acciones que se realizan sobre un sistema con el fin de obtener un resultado específico. Las mismas representan las interacciones entre el actor y el sistema, y son \textquote{piezas de construcción reutilizables} para elaborar los casos de uso. Definir estas operaciones de forma previa al desarrollo nos va a permitir identificar patrones comunes en el funcionamiento interno de nuestra plataforma, ayudándonos a hacerla más modular y fácil de testear de forma unitaria.

\begin{enumerate}[label=SOP-\protect\twodigits{\arabic*}:, align=left, leftmargin=*]
        
% Ver lista de pedidos
\item \textbf{Obtener lista de pedidos}
\begin{itemize}
\item \textbf{Nombre}: retrieveOrderList()
\item \textbf{Descripción}: Permite obtener una lista de todos los pedidos realizados en la plataforma.
\item \textbf{Parámetros de entrada}: Ninguno
\item \textbf{Parámetros de salida}:
\begin{itemize}
\item Lista de pedidos, con información detallada de cada uno.
\end{itemize}
\end{itemize}

% Ver lista de usuarios
\item \textbf{Obtener lista de usuarios}
\begin{itemize}
\item \textbf{Nombre}: retrieveUserList()
\item \textbf{Descripción}: Permite obtener una lista de todos los usuarios registrados en la plataforma.
\item \textbf{Parámetros de entrada}: Ninguno
\item \textbf{Parámetros de salida}:
\begin{itemize}
\item Lista de usuarios registrados, con información relevante de cada uno.
\end{itemize}
\end{itemize}

% Ver lista de denuncias
\item \textbf{Obtener lista de denuncias}
\begin{itemize}
\item \textbf{Nombre}: retrieveReportList()
\item \textbf{Descripción}: Permite obtener una lista de todas las denuncias registradas en la plataforma.
\item \textbf{Parámetros de entrada}: Ninguno
\item \textbf{Parámetros de salida}:
\begin{itemize}
\item Lista de denuncias, con información relevante de cada una.
\end{itemize}
\end{itemize}

\item \textbf{Resolver denuncia}
\begin{itemize}
\item \textbf{Nombre}: resolveReport(reportId, actions)
\item \textbf{Descripción}: Permite al administrador resolver una denuncia tomando las acciones necesarias.
\item \textbf{Parámetros de entrada}:
\begin{itemize}
\item Identificador único de la denuncia
\item Acciones tomadas por el administrador para resolver la denuncia
\end{itemize}
\item \textbf{Parámetros de salida}: Ninguno
\end{itemize}

% Operaciones asociadas
\item \textbf{Obtener lista de publicaciones}
\begin{itemize}
\item \textbf{Nombre}: retrieveListingList()
\item \textbf{Descripción}: Permite obtener una lista de todas las publicaciones realizadas en la plataforma.
\item \textbf{Parámetros de entrada}: Ninguno
\item \textbf{Parámetros de salida}:
\begin{itemize}
\item Lista de publicaciones
\end{itemize}
\end{itemize}

\item \textbf{Eliminar publicación}
\begin{itemize}
\item \textbf{Nombre}: deleteListing(listingId)
\item \textbf{Descripción}: Permite eliminar una publicación de la plataforma.
\item \textbf{Parámetros de entrada}:
\begin{itemize}
\item Identificador único de la publicación a eliminar
\end{itemize}
\item \textbf{Parámetros de salida}: Ninguno
\end{itemize}

% Desactivar usuario
\item \textbf{Desactivar cuenta de usuario}
\begin{itemize}
\item \textbf{Nombre}: deactivateUser(userId)
\item \textbf{Descripción}: Permite desactivar la cuenta de un usuario en la plataforma.
\item \textbf{Parámetros de entrada}:
\begin{itemize}
\item Identificador único del usuario cuya cuenta se desea desactivar.
\end{itemize}
\item \textbf{Parámetros de salida}: Ninguno
\end{itemize}

% Ver estadísticas y análisis de la plataforma
\item \textbf{Obtener estadísticas de la plataforma}
\begin{itemize}
\item \textbf{Nombre}: retrievePlatformStatistics()
\item \textbf{Descripción}: Permite obtener las estadísticas de la plataforma, como el número total de publicaciones, el número de publicaciones activas y el número de usuarios registrados.
\item \textbf{Parámetros de entrada}: Ninguno.
\item \textbf{Parámetros de salida}:
\begin{itemize}
\item Estadísticas de la plataforma (número total de publicaciones, número de publicaciones activas, número de usuarios registrados, etc.).
\end{itemize}
\end{itemize}

\item \textbf{Obtener análisis de la plataforma}
\begin{itemize}
\item \textbf{Nombre}: retrievePlatformAnalysis()
\item \textbf{Descripción}: Permite obtener análisis de la plataforma para identificar patrones, tendencias o áreas de mejora.
\item \textbf{Parámetros de entrada}: Ninguno.
\item \textbf{Parámetros de salida}:
\begin{itemize}
\item Análisis de la plataforma (patrones de uso, tendencias de actividad, áreas de mejora, etc.).
\end{itemize}
\end{itemize}

% Enviar notificación personalizada
\item \textbf{Enviar notificación personalizada}
\begin{itemize}
\item \textbf{Nombre}: sendCustomNotification(content, recipients)
\item \textbf{Descripción}: Permite enviar notificaciones personalizadas a los usuarios de la plataforma.
\item \textbf{Parámetros de entrada}:
\begin{itemize}
\item Contenido: el contenido de la notificación.
\item Destinatarios: una lista de usuarios destinatarios a los cuales se enviará la notificación.
\end{itemize}
\item \textbf{Parámetros de salida}: Ninguno
\end{itemize}

% Registrarme mediante correo electrónico
\item \textbf{Registrar usuario}
\begin{itemize}
\item \textbf{Nombre}: registerUser(email, password)
\item \textbf{Descripción}: Permite registrar un nuevo usuario en la aplicación utilizando una dirección de correo electrónico y una contraseña.
\item \textbf{Parámetros de entrada}:
\begin{itemize}
\item Dirección de correo electrónico del usuario.
\item Contraseña del usuario.
\end{itemize}
\item \textbf{Parámetros de salida}: Ninguno.
\end{itemize}

\item \textbf{Verificar registro}
\begin{itemize}
\item \textbf{Nombre}: verifyRegistration(email)
\item \textbf{Descripción}: Permite verificar el registro de un usuario mediante la confirmación de su dirección de correo electrónico.
\item \textbf{Parámetros de entrada}:
\begin{itemize}
\item Dirección de correo electrónico del usuario.
\end{itemize}
\item \textbf{Parámetros de salida}: Ninguno.
\end{itemize}

\item \textbf{Reenviar correo de verificación}
\begin{itemize}
\item \textbf{Nombre}: resendVerificationEmail(email)
\item \textbf{Descripción}: Permite reenviar el correo electrónico de verificación a un usuario que no ha recibido el correo electrónico inicial.
\item \textbf{Parámetros de entrada}:
\begin{itemize}
\item Dirección de correo electrónico del usuario.
\end{itemize}
\item \textbf{Parámetros de salida}: Ninguno.
\end{itemize}

% Iniciar sesión mediante correo electrónico
\item \textbf{Iniciar sesión}
\begin{itemize}
\item \textbf{Nombre}: loginUser(email, password)
\item \textbf{Descripción}: Permite al usuario iniciar sesión en la aplicación utilizando su correo electrónico y contraseña.
\item \textbf{Parámetros de entrada}:
\begin{itemize}
\item Dirección de correo electrónico del usuario.
\item Contraseña del usuario.
\end{itemize}
\item \textbf{Parámetros de salida}: Token de sesión.
\end{itemize}

% Inicio de sesión mediante OTP
\item \textbf{Generar OTP}
\begin{itemize}
\item \textbf{Nombre}: generateOTP(phoneNumber)
\item \textbf{Descripción}: Genera una OTP única y la asocia con el número de teléfono móvil proporcionado.
\item \textbf{Parámetros de entrada}:
\begin{itemize}
\item Número de teléfono móvil del usuario
\end{itemize}
\item \textbf{Parámetros de salida}:
\begin{itemize}
\item OTP generada
\end{itemize}
\end{itemize}

\item \textbf{Enviar OTP mediante SMS}
\begin{itemize}
\item \textbf{Nombre}: sendOTPviaSMS(phoneNumber, otp)
\item \textbf{Descripción}: Envía la OTP al número de teléfono móvil del usuario mediante un mensaje de texto (SMS).
\item \textbf{Parámetros de entrada}:
\begin{itemize}
\item Número de teléfono móvil del usuario
\item OTP generada
\end{itemize}
\item \textbf{Parámetros de salida}: Ninguno
\end{itemize}

\item \textbf{Verificar OTP}
\begin{itemize}
\item \textbf{Nombre}: verifyOTP(phoneNumber, otp)
\item \textbf{Descripción}: Verifica si la OTP ingresada por el usuario coincide con la OTP generada previamente.
\item \textbf{Parámetros de entrada}:
\begin{itemize}
\item Número de teléfono móvil del usuario
\item OTP ingresada
\end{itemize}
\item \textbf{Parámetros de salida}:
\begin{itemize}
\item Resultado de verificación (coincide o no coincide)
\end{itemize}
\end{itemize}

\item \textbf{Iniciar sesión mediante OTP}
\begin{itemize}
\item \textbf{Nombre}: loginOTP(phoneNumber, otp)
\item \textbf{Descripción}: Permite al usuario no autenticado iniciar sesión en la aplicación utilizando su número de teléfono móvil y la OTP generada.
\item \textbf{Parámetros de entrada}:
\begin{itemize}
\item Número de teléfono móvil del usuario
\item OTP generada
\end{itemize}
\item \textbf{Parámetros de salida}:
\begin{itemize}
\item Token de sesión
\end{itemize}
\end{itemize}

% Restablecer mi contraseña
\item \textbf{Verificar dirección de correo electrónico}
\begin{itemize}
\item \textbf{Nombre}: verifyEmail(email)
\item \textbf{Descripción}: Permite al sistema verificar si una dirección de correo electrónico es válida y está asociada a una cuenta de usuario.
\item \textbf{Parámetros de entrada}:
\begin{itemize}
\item Dirección de correo electrónico a verificar
\end{itemize}
\item \textbf{Parámetros de salida}:
\begin{itemize}
\item Booleano que indica si la dirección de correo electrónico es válida y está asociada a una cuenta de usuario
\end{itemize}
\end{itemize}

\item \textbf{Generar enlace de restablecimiento de contraseña}
\begin{itemize}
\item \textbf{Nombre}: generateResetPasswordLink(email)
\item \textbf{Descripción}: Permite al sistema generar un enlace único de restablecimiento de contraseña para una dirección de correo electrónico asociada a una cuenta de usuario.
\item \textbf{Parámetros de entrada}:
\begin{itemize}
\item Dirección de correo electrónico asociada a la cuenta de usuario
\end{itemize}
\item \textbf{Parámetros de salida}:
\begin{itemize}
\item Enlace de restablecimiento de contraseña único
\end{itemize}
\end{itemize}

\item \textbf{Actualizar contraseña de usuario}
\begin{itemize}
\item \textbf{Nombre}: updatePassword(token, newPassword)
\item \textbf{Descripción}: Permite al usuario actualizar su contraseña.
\item \textbf{Parámetros de entrada}:
\begin{itemize}
\item Token de restablecimiento de contraseña
\item Nueva contraseña
\end{itemize}
\item \textbf{Parámetros de salida}: Ninguno
\end{itemize}

% Ver lista de últimas publicaciones
\item \textbf{Obtener últimas publicaciones}
\begin{itemize}
\item \textbf{Nombre}: retrieveLatestListings()
\item \textbf{Descripción}: Permite obtener la lista de las últimas publicaciones realizadas en la aplicación.
\item \textbf{Parámetros de entrada}: Ninguno
\item \textbf{Parámetros de salida}:
\begin{itemize}
\item Lista de publicaciones (título, imagen, precio, entre otros campos)
\end{itemize}
\end{itemize}

% Ver lista de publicaciones cercanas
\item \textbf{Obtener ubicación del usuario}
\begin{itemize}
\item \textbf{Nombre}: retrieveLocation()
\item \textbf{Descripción}: Permite obtener la ubicación del usuario.
\item \textbf{Parámetros de entrada}: Ninguno
\item \textbf{Parámetros de salida}:
\begin{itemize}
\item Ubicación del usuario (latitud, longitud)
\end{itemize}
\end{itemize}

\item \textbf{Obtener lista de publicaciones cercanas}
\begin{itemize}
\item \textbf{Nombre}: retrieveNearbyListings(location)
\item \textbf{Descripción}: Permite obtener una lista de publicaciones cercanas a una ubicación dada.
\item \textbf{Parámetros de entrada}:
\begin{itemize}
\item Ubicación del usuario (latitud, longitud)
\end{itemize}
\item \textbf{Parámetros de salida}:
\begin{itemize}
\item Lista de publicaciones ordenadas por distancia
\end{itemize}
\end{itemize}

% Ver publicaciones seleccionadas
\item \textbf{Consultar publicaciones seleccionadas}
\begin{itemize}
\item \textbf{Nombre}: retrieveSelectedListings()
\item \textbf{Descripción}: Permite obtener la lista de publicaciones seleccionadas por el equipo de la plataforma.
\item \textbf{Parámetros de entrada}: Ninguno
\item \textbf{Parámetros de salida}:
\begin{itemize}
\item Lista de publicaciones seleccionadas
\end{itemize}
\end{itemize}

% Ver productores en un mapa
\item \textbf{Consultar ubicación de productores}
\begin{itemize}
\item \textbf{Nombre}: retrieveProducersLocation()
\item \textbf{Descripción}: Permite obtener la información de ubicación de los productores registrados en la plataforma.
\item \textbf{Parámetros de entrada}: Ninguno
\item \textbf{Parámetros de salida}:
\begin{itemize}
\item Lista de ubicaciones de los productores
\end{itemize}
\end{itemize}

% Ver detalles de publicación
\item \textbf{Consultar detalles de publicación}
\begin{itemize}
\item \textbf{Nombre}: retrieveListing(listingId)
\item \textbf{Descripción}: Permite consultar los detalles de una publicación específica.
\item \textbf{Parámetros de entrada}:
\begin{itemize}
\item Identificador único de la publicación
\end{itemize}
\item \textbf{Parámetros de salida}:
\begin{itemize}
\item Detalles de la publicación (título, descripción, imágenes, precio, cantidad disponible, etc.)
\end{itemize}
\end{itemize}

% Ver perfil de productor
\item \textbf{Consultar perfil de productor}
\begin{itemize}
\item \textbf{Nombre}: retrieveProducerProfile(producerId)
\item \textbf{Descripción}: Permite consultar el perfil de un productor.
\item \textbf{Parámetros de entrada}:
\begin{itemize}
\item Identificador único del productor
\end{itemize}
\item \textbf{Parámetros de salida}:
\begin{itemize}
\item Información del perfil del productor (nombre, biografía, ubicación, imagen de perfil, lista de productos)
\end{itemize}
\end{itemize}

% Ver lista de reseñas de productor
\item \textbf{Consultar reseñas de productor}
\begin{itemize}
\item \textbf{Nombre}: retrieveProducerReviews(producerId)
\item \textbf{Descripción}: Permite consultar las reseñas de un productor en particular.
\item \textbf{Parámetros de entrada}:
\begin{itemize}
\item Identificador único del productor
\end{itemize}
\item \textbf{Parámetros de salida}:
\begin{itemize}
\item Lista de reseñas del productor
\end{itemize}
\end{itemize}

% Buscar publicaciones
\item \textbf{Buscar publicaciones}
\begin{itemize}
\item \textbf{Nombre}: searchListings(query, filters)
\item \textbf{Descripción}: Permite buscar publicaciones en la plataforma utilizando palabras clave y filtros.
\item \textbf{Parámetros de entrada}:
\begin{itemize}
\item Palabras clave (query): Cadena de texto que contiene las palabras clave relacionadas con el producto que se desea buscar.
\item Filtros (filters): Conjunto de filtros opcionales para refinar la búsqueda. Incluye parámetros como precio mínimo, precio máximo, cantidad disponible mínima, productor (id), alérgenos (lista separada por comas), características (lista separada por comas) y distancia (en metros). Además, se pueden especificar opciones de ordenamiento, como ordenar por precio, cantidad disponible y fecha de creación.
\end{itemize}
\item \textbf{Parámetros de salida}:
\begin{itemize}
\item Lista de publicaciones que coinciden con las palabras clave y los filtros aplicados.
\end{itemize}
\end{itemize}

% Ver mi perfil privado
\item \textbf{Consultar datos de perfil}
\begin{itemize}
  \item \itembf{Nombre}: retrieveUser(userId)
\item \textbf{Descripción}: Permite consultar todos los datos de un perfil de usuario.
\item \textbf{Parámetros de entrada}:
\begin{itemize}
\item Identificador único del perfil de usuario
\end{itemize}
\item \textbf{Parámetros de salida}:
\begin{itemize}
\item Datos del perfil de usuario
\end{itemize}
\end{itemize}

% Editar mi perfil
\item \textbf{Editar datos de perfil}
\begin{itemize}
  \item \itembf{Nombre}: updateUser(userId, newData)
\item \textbf{Descripción}: Permite editar los datos de un perfil de usuario.
\item \textbf{Parámetros de entrada}:
\begin{itemize}
\item Identificador único del perfil de usuario
\item Datos del perfil de usuario a actualizar
\end{itemize}
\item \textbf{Parámetros de salida}:
\begin{itemize}
\item Datos del perfil de usuario actualizados
\end{itemize}
\end{itemize}

% Eliminar mi cuenta
\item \textbf{Eliminar cuenta de usuario}
\begin{itemize}
\item \textbf{Nombre}: deleteUser(userId)
\item \textbf{Descripción}: Permite eliminar una cuenta de usuario y anonimizar sus datos personales.
\item \textbf{Parámetros de entrada}:
\begin{itemize}
\item Identificador único del usuario
\end{itemize}
\item \textbf{Parámetros de salida}:
\begin{itemize}
\item Mensaje de confirmación de eliminación de cuenta
\end{itemize}
\end{itemize}

% Ver lista de mis notificaciones
\item \textbf{Consultar notificaciones de usuario}
\begin{itemize}
\item \textbf{Nombre}: retrieveUserNotifications(userId)
\item \textbf{Descripción}: Permite consultar la lista de notificaciones de un usuario.
\item \textbf{Parámetros de entrada}:
\begin{itemize}
\item Identificador único del usuario
\end{itemize}
\item \textbf{Parámetros de salida}:
\begin{itemize}
\item Lista de notificaciones del usuario
\end{itemize}
\end{itemize}

% Ver mi historial de compras
\item \textbf{Consultar historial de compras}
\begin{itemize}
\item \textbf{Nombre}: retrievePurchaseHistory()
\item \textbf{Descripción}: Permite consultar el historial de compras del usuario.
\item \textbf{Parámetros de entrada}: Ninguno
\item \textbf{Parámetros de salida}:
\begin{itemize}
\item Lista de compras realizadas por el usuario, ordenadas por fecha y agrupadas por año y mes
\end{itemize}
\end{itemize}

% Valorar compra
\item \textbf{Consultar detalles de compra}
\begin{itemize}
\item \textbf{Nombre}: retrievePurchaseDetails(purchaseId)
\item \textbf{Descripción}: Permite consultar los detalles de una compra realizada.
\item \textbf{Parámetros de entrada}:
\begin{itemize}
\item Identificador único de la compra
\end{itemize}
\item \textbf{Parámetros de salida}:
\begin{itemize}
\item Detalles de la compra
\end{itemize}
\end{itemize}

\item \textbf{Valorar compra}
\begin{itemize}
\item \textbf{Nombre}: ratePurchase(purchaseId, rating, comment)
\item \textbf{Descripción}: Permite valorar una compra realizada.
\item \textbf{Parámetros de entrada}:
\begin{itemize}
\item Identificador único de la compra
\item Puntuación de 1 a 5 estrellas
\item Comentario adicional (opcional)
\end{itemize}
\item \textbf{Parámetros de salida}: Ninguno
\end{itemize}

% Establecer preferencias de notificaciones
\item \textbf{Obtener preferencias de notificaciones}
\begin{itemize}
\item \textbf{Nombre}: retrieveNotificationPreferences(userId)
\item \textbf{Descripción}: Permite obtener las preferencias de notificaciones de un usuario.
\item \textbf{Parámetros de entrada}:
\begin{itemize}
\item Identificador único del usuario
\end{itemize}
\item \textbf{Parámetros de salida}:
\begin{itemize}
\item Tipo de notificaciones configuradas por el usuario
\item Métodos de comunicación configurados por el usuario
\end{itemize}
\end{itemize}

\item \textbf{Establecer preferencias de notificaciones}
\begin{itemize}
\item \textbf{Nombre}: setNotificationPreferences(userId, notificationType, communicationMethods)
\item \textbf{Descripción}: Permite al usuario establecer sus preferencias de notificaciones.
\item \textbf{Parámetros de entrada}:
\begin{itemize}
\item Identificador único del usuario
\item Tipo de notificaciones (actualizaciones de productos, promociones, noticias, etc.)
\item Métodos de comunicación (in-app, push, correo electrónico, SMS)
\end{itemize}
\item \textbf{Parámetros de salida}: Ninguno
\end{itemize}

% Publicar productos en venta
\item \textbf{Crear publicación de producto}
\begin{itemize}
\item \textbf{Nombre}: createListing(productData)
\item \textbf{Descripción}: Permite crear una nueva publicación de producto en la plataforma.
\item \textbf{Parámetros de entrada}:
\begin{itemize}
\item Datos del producto (título, descripción, imágenes, cantidad disponible, precio, alérgenos, características, etc.)
\end{itemize}
\item \textbf{Parámetros de salida}:
\begin{itemize}
\item Identificador único de la publicación
\end{itemize}
\end{itemize}

% Editar publicación
\item \textbf{Editar publicación}
\begin{itemize}
\item \textbf{Nombre}: updateListing(listingId, listingData)
\item \textbf{Descripción}: Permite editar los campos de una publicación existente.
\item \textbf{Parámetros de entrada}:
\begin{itemize}
\item Identificador único de la publicación
\item Datos modificados de la publicación
\end{itemize}
\item \textbf{Parámetros de salida}:
\begin{itemize}
\item Confirmación de que los cambios se han guardado con éxito
\end{itemize}
\end{itemize}

% Marcar publicación como vendida a un usuario
\item \textbf{Marcar publicación como vendida a un usuario}
\begin{itemize}
\item \textbf{Nombre}: markListingAsSoldToUser(listingId, buyerId, quantity, totalPrice)
\item \textbf{Descripción}: Permite marcar una publicación como vendida a un usuario y registrar la transacción.
\item \textbf{Parámetros de entrada}:
\begin{itemize}
\item Identificador único de la publicación
\item Identificador único del usuario comprador
\item Cantidad de producto vendida
\item Precio total de la venta
\end{itemize}
\item \textbf{Parámetros de salida}: Ninguno
\end{itemize}

\item \textbf{Enviar notificación de valoración}
\begin{itemize}
\item \textbf{Nombre}: sendRatingNotification(buyerId)
\item \textbf{Descripción}: Envía una notificación al comprador solicitando que valore la compra realizada.
\item \textbf{Parámetros de entrada}:
\begin{itemize}
\item Identificador único del comprador
\end{itemize}
\item \textbf{Parámetros de salida}: Ninguno
\end{itemize}

% Marcar publicación como vendida fuera de la plataforma
\item \textbf{Marcar publicación como vendida fuera de la plataforma}
\begin{itemize}
\item \textbf{Nombre}: markListingAsSoldOutside(listingId, quantitySold, totalPrice)
\item \textbf{Descripción}: Permite marcar una publicación como vendida y registrar la venta.
\item \textbf{Parámetros de entrada}:
\begin{itemize}
\item Identificador único de la publicación
\item Cantidad de producto vendida
\item Precio total de la venta
\end{itemize}
\item \textbf{Parámetros de salida}: Ninguno
\end{itemize}

% Desactivar publicación
\item \textbf{Desactivar publicación}
\begin{itemize}
\item \textbf{Nombre}: setListingEnableStauts(listingId, enabled)
\item \textbf{Descripción}: Permite activar o desactivar una publicación.
\item \textbf{Parámetros de entrada}:
\begin{itemize}
\item Identificador único de la publicación a desactivar
\item Estado deseado de la publicación (activa o desactivada)
\end{itemize}
\item \textbf{Parámetros de salida}: Ninguno
\end{itemize}

% Eliminar publicación
\item \textbf{Eliminar publicación}
\begin{itemize}
\item \textbf{Nombre}: deleteListing(listingId)
\item \textbf{Descripción}: Permite eliminar una publicación de la plataforma.
\item \textbf{Parámetros de entrada}:
\begin{itemize}
\item Identificador único de la publicación
\end{itemize}
\end{itemize}

% Ver mi historial de ventas
\item \textbf{Consultar historial de ventas del productor}
\begin{itemize}
\item \textbf{Nombre}: retrieveSalesHistory(producerId)
\item \textbf{Descripción}: Permite consultar el historial de ventas realizadas por un productor.
\item \textbf{Parámetros de entrada}:
\begin{itemize}
\item Identificador único del productor
\end{itemize}
\item \textbf{Parámetros de salida}:
\begin{itemize}
\item Lista de ventas realizadas, ordenadas por fecha de compra y agrupadas por año y mes
\end{itemize}
\end{itemize}

% Responder mensajes
\item \textbf{Consultar mensajes de usuario}
\begin{itemize}
\item \textbf{Nombre}: retrieveMessages(conversationId)
\item \textbf{Descripción}: Permite consultar los mensajes de un usuario en una conversación específica.
\item \textbf{Parámetros de entrada}:
\begin{itemize}
\item Identificador único de la conversación
\end{itemize}
\item \textbf{Parámetros de salida}:
\begin{itemize}
\item Lista de mensajes de usuario
\end{itemize}
\end{itemize}

\item \textbf{Enviar mensaje}
\begin{itemize}
\item \textbf{Nombre}: sendMessage(conversationId, messageContent)
\item \textbf{Descripción}: Permite enviar un mensaje como respuesta a un usuario en una conversación específica.
\item \textbf{Parámetros de entrada}:
\begin{itemize}
\item Identificador único de la conversación
\item Contenido del mensaje
\end{itemize}
\item \textbf{Parámetros de salida}: Ninguno
\end{itemize}

% Exportar mi historial de pedidos a Excel
\item \textbf{Exportar historial de pedidos a Excel}
\begin{itemize}
\item \textbf{Nombre}: exportOrdersAsExcel()
\item \textbf{Descripción}: Permite exportar el historial de pedidos del productor en formato Excel.
\item \textbf{Parámetros de entrada}:
\begin{itemize}
\item Identificador único del productor
\end{itemize}
\item \textbf{Parámetros de salida}:
\begin{itemize}
\item Archivo Excel con el historial de pedidos del productor
\end{itemize}
\end{itemize}

\end{enumerate}

% !TEX root = ../../proyect.tex

\section{Matrices de trazabilidad}\label{sec:matrices-trazabilidad}
\todo{Volver a exportar las tres matrices, he quitado el OBJ-01 (admin)}
\subsection{Matriz OBJ-IRQ}
\begin{table}[!ht]
    \centering
    \small
    \begin{tabular}{|c|c|c|c|c|c|c|c|c|c|}
    \hline
        ~ & OBJ-01 & OBJ-02 & OBJ-03 & OBJ-04 & OBJ-05 & OBJ-06 & OBJ-07 & OBJ-08 & OBJ-09 \\ \hline
        IRQ-01 & X & ~ & ~ & ~ & ~ & ~ & ~ & ~ & ~ \\ \hline
        IRQ-02 & ~ & X & ~ & ~ & ~ & ~ & ~ & ~ & ~ \\ \hline
        IRQ-03 & ~ & X & ~ & ~ & ~ & ~ & ~ & ~ & ~ \\ \hline
        IRQ-04 & ~ & ~ & X & ~ & ~ & ~ & ~ & ~ & ~ \\ \hline
        IRQ-05 & ~ & ~ & ~ & X & ~ & ~ & ~ & ~ & ~ \\ \hline
        IRQ-06 & ~ & ~ & ~ & ~ & X & ~ & ~ & ~ & ~ \\ \hline
        IRQ-07 & ~ & ~ & ~ & ~ & ~ & X & ~ & ~ & ~ \\ \hline
        IRQ-08 & ~ & ~ & ~ & ~ & ~ & ~ & ~ & X & ~ \\ \hline
        IRQ-09 & ~ & ~ & ~ & ~ & ~ & ~ & ~ & X & ~ \\ \hline
    \end{tabular}
\end{table}

\newpage
\subsection{Matriz OBJ-UC}
\begin{table}[!ht]
    \centering
    \small
    \begin{tabular}{|c|c|c|c|c|c|c|c|c|c|}
    \hline
        ~ & OBJ-01 & OBJ-02 & OBJ-03 & OBJ-04 & OBJ-05 & OBJ-06 & OBJ-07 & OBJ-08 & OBJ-09 \\ \hline
        UC-01 & X & ~ & ~ & X & ~ & ~ & ~ & ~ & ~ \\ \hline
        UC-02 & X & X & ~ & ~ & ~ & ~ & ~ & ~ & ~ \\ \hline
        UC-03 & X & ~ & ~ & ~ & ~ & X & ~ & ~ & ~ \\ \hline
        UC-04 & X & ~ & X & ~ & ~ & ~ & ~ & ~ & ~ \\ \hline
        UC-05 & X & X & ~ & ~ & ~ & ~ & ~ & ~ & ~ \\ \hline
        UC-06 & X & ~ & ~ & ~ & ~ & ~ & ~ & ~ & ~ \\ \hline
        UC-07 & X & ~ & ~ & ~ & ~ & ~ & ~ & ~ & X \\ \hline
        UC-08 & ~ & X & ~ & ~ & ~ & ~ & ~ & ~ & ~ \\ \hline
        UC-09 & ~ & X & ~ & ~ & ~ & ~ & ~ & ~ & ~ \\ \hline
        UC-10 & ~ & X & ~ & ~ & ~ & ~ & ~ & ~ & ~ \\ \hline
        UC-11 & ~ & X & ~ & ~ & ~ & ~ & ~ & ~ & ~ \\ \hline
        UC-12 & ~ & X & ~ & ~ & ~ & ~ & ~ & ~ & ~ \\ \hline
        UC-13 & ~ & X & ~ & ~ & ~ & ~ & ~ & ~ & ~ \\ \hline
        UC-14 & ~ & X & ~ & ~ & ~ & ~ & ~ & ~ & ~ \\ \hline
        UC-15 & ~ & ~ & X & ~ & ~ & ~ & ~ & ~ & ~ \\ \hline
        UC-16 & ~ & ~ & X & ~ & ~ & ~ & ~ & ~ & ~ \\ \hline
        UC-17 & ~ & ~ & X & ~ & ~ & ~ & ~ & ~ & ~ \\ \hline
        UC-18 & ~ & X & X & ~ & ~ & ~ & ~ & ~ & ~ \\ \hline
        UC-19 & ~ & ~ & X & ~ & ~ & ~ & ~ & ~ & ~ \\ \hline
        UC-20 & ~ & X & ~ & ~ & ~ & ~ & ~ & ~ & ~ \\ \hline
        UC-21 & ~ & X & ~ & ~ & X & ~ & ~ & ~ & ~ \\ \hline
        UC-22 & ~ & ~ & X & ~ & ~ & ~ & ~ & ~ & ~ \\ \hline
        UC-23 & ~ & ~ & X & ~ & ~ & ~ & ~ & ~ & ~ \\ \hline
        UC-24 & ~ & ~ & X & ~ & ~ & ~ & ~ & ~ & ~ \\ \hline
        UC-25 & ~ & ~ & X & ~ & ~ & ~ & ~ & ~ & ~ \\ \hline
        UC-26 & ~ & ~ & X & ~ & ~ & ~ & ~ & ~ & ~ \\ \hline
        UC-27 & ~ & X & ~ & ~ & ~ & ~ & ~ & ~ & ~ \\ \hline
        UC-28 & ~ & X & ~ & ~ & ~ & ~ & ~ & ~ & ~ \\ \hline
        UC-29 & ~ & X & ~ & ~ & ~ & ~ & ~ & ~ & ~ \\ \hline
        UC-30 & ~ & ~ & ~ & ~ & ~ & ~ & ~ & ~ & X \\ \hline
        UC-31 & ~ & ~ & ~ & X & ~ & ~ & ~ & ~ & ~ \\ \hline
        UC-32 & ~ & X & ~ & X & X & ~ & ~ & ~ & ~ \\ \hline
        UC-33 & ~ & ~ & ~ & ~ & ~ & ~ & ~ & ~ & X \\ \hline
        UC-34 & ~ & ~ & X & ~ & ~ & ~ & ~ & ~ & ~ \\ \hline
        UC-35 & ~ & ~ & X & ~ & ~ & ~ & ~ & ~ & ~ \\ \hline
        UC-36 & ~ & X & X & X & ~ & ~ & ~ & ~ & ~ \\ \hline
        UC-37 & ~ & ~ & X & ~ & ~ & ~ & ~ & ~ & ~ \\ \hline
        UC-38 & ~ & ~ & X & ~ & ~ & ~ & ~ & ~ & ~ \\ \hline
        UC-39 & ~ & X & ~ & X & ~ & ~ & ~ & ~ & ~ \\ \hline
        UC-40 & ~ & X & ~ & X & ~ & ~ & ~ & ~ & ~ \\ \hline
        UC-41 & ~ & ~ & ~ & ~ & ~ & ~ & ~ & X & ~ \\ \hline
        UC-42 & ~ & ~ & ~ & X & ~ & ~ & ~ & ~ & ~ \\ \hline
    \end{tabular}
\end{table}

\newpage
\subsection{Matriz OBJ-HU}
\begin{table}[!ht]
    \centering
    \scriptsize
    \begin{tabular}{|c|c|c|c|c|c|c|c|c|c|}
    \hline
        ~ & OBJ-01 & OBJ-02 & OBJ-03 & OBJ-04 & OBJ-05 & OBJ-06 & OBJ-07 & OBJ-08 & OBJ-09 \\ \hline
        HU-01 & X & ~ & ~ & X & ~ & ~ & ~ & ~ & ~ \\ \hline
        HU-02 & X & X & ~ & ~ & ~ & ~ & ~ & ~ & ~ \\ \hline
        HU-03 & X & ~ & ~ & ~ & ~ & X & ~ & ~ & ~ \\ \hline
        HU-04 & X & ~ & X & ~ & ~ & ~ & ~ & ~ & ~ \\ \hline
        HU-05 & X & X & ~ & ~ & ~ & ~ & ~ & ~ & ~ \\ \hline
        HU-06 & X & ~ & ~ & ~ & ~ & ~ & ~ & ~ & ~ \\ \hline
        HU-07 & X & X & ~ & ~ & ~ & ~ & ~ & ~ & ~ \\ \hline
        HU-08 & X & X & ~ & ~ & ~ & ~ & ~ & ~ & ~ \\ \hline
        HU-09 & ~ & ~ & ~ & ~ & ~ & ~ & X & ~ & ~ \\ \hline
        HU-10 & ~ & ~ & ~ & ~ & ~ & ~ & X & ~ & ~ \\ \hline
        HU-11 & ~ & ~ & ~ & ~ & ~ & ~ & X & ~ & ~ \\ \hline
        HU-12 & ~ & ~ & ~ & ~ & ~ & ~ & X & ~ & ~ \\ \hline
        HU-13 & ~ & ~ & ~ & ~ & ~ & ~ & X & ~ & ~ \\ \hline
        HU-14 & ~ & X & ~ & ~ & ~ & ~ & ~ & ~ & ~ \\ \hline
        HU-15 & ~ & X & ~ & ~ & ~ & ~ & ~ & ~ & ~ \\ \hline
        HU-16 & ~ & X & ~ & ~ & ~ & ~ & ~ & ~ & ~ \\ \hline
        HU-17 & ~ & X & ~ & ~ & ~ & ~ & ~ & ~ & ~ \\ \hline
        HU-18 & ~ & X & ~ & ~ & ~ & ~ & ~ & ~ & ~ \\ \hline
        HU-19 & ~ & X & ~ & ~ & ~ & ~ & ~ & ~ & ~ \\ \hline
        HU-20 & ~ & X & ~ & ~ & ~ & ~ & ~ & ~ & ~ \\ \hline
        HU-21 & ~ & ~ & X & ~ & ~ & ~ & ~ & ~ & ~ \\ \hline
        HU-22 & ~ & ~ & X & ~ & ~ & ~ & ~ & ~ & ~ \\ \hline
        HU-23 & ~ & X & X & ~ & ~ & ~ & ~ & ~ & ~ \\ \hline
        HU-24 & ~ & ~ & X & ~ & ~ & ~ & ~ & ~ & ~ \\ \hline
        HU-25 & ~ & ~ & X & ~ & ~ & ~ & ~ & ~ & ~ \\ \hline
        HU-26 & ~ & X & ~ & ~ & ~ & ~ & ~ & ~ & ~ \\ \hline
        HU-27 & ~ & X & ~ & ~ & X & ~ & ~ & ~ & ~ \\ \hline
        HU-28 & ~ & ~ & X & ~ & ~ & ~ & ~ & ~ & ~ \\ \hline
        HU-29 & ~ & ~ & X & ~ & ~ & ~ & ~ & ~ & ~ \\ \hline
        HU-30 & ~ & ~ & X & ~ & ~ & ~ & ~ & ~ & ~ \\ \hline
        HU-31 & ~ & ~ & X & ~ & ~ & ~ & ~ & ~ & ~ \\ \hline
        HU-32 & ~ & ~ & X & ~ & ~ & ~ & ~ & ~ & ~ \\ \hline
        HU-33 & ~ & X & ~ & ~ & ~ & ~ & ~ & ~ & ~ \\ \hline
        HU-34 & ~ & X & ~ & ~ & ~ & ~ & ~ & ~ & ~ \\ \hline
        HU-35 & ~ & X & ~ & ~ & ~ & ~ & ~ & ~ & ~ \\ \hline
        HU-36 & ~ & ~ & ~ & ~ & ~ & ~ & ~ & ~ & X \\ \hline
        HU-37 & ~ & ~ & ~ & ~ & ~ & ~ & ~ & ~ & X \\ \hline
        HU-38 & ~ & ~ & X & ~ & ~ & ~ & ~ & ~ & ~ \\ \hline
        HU-39 & ~ & ~ & X & ~ & ~ & ~ & ~ & ~ & ~ \\ \hline
        HU-40 & ~ & ~ & X & ~ & ~ & ~ & ~ & ~ & X \\ \hline
        HU-41 & ~ & ~ & X & ~ & ~ & ~ & ~ & ~ & ~ \\ \hline
        HU-42 & ~ & X & ~ & ~ & ~ & X & ~ & ~ & ~ \\ \hline
        HU-43 & ~ & X & ~ & X & ~ & X & ~ & ~ & ~ \\ \hline
        HU-44 & ~ & X & ~ & ~ & ~ & ~ & ~ & X & ~ \\ \hline
        HU-45 & ~ & ~ & ~ & X & ~ & ~ & ~ & ~ & ~ \\ \hline
        HU-46 & ~ & ~ & ~ & ~ & X & ~ & ~ & ~ & ~ \\ \hline
        HU-47 & ~ & ~ & ~ & ~ & X & ~ & ~ & ~ & ~ \\ \hline
        HU-48 & ~ & X & ~ & ~ & ~ & ~ & ~ & ~ & ~ \\ \hline
        HU-49 & ~ & ~ & ~ & ~ & ~ & ~ & ~ & ~ & X \\ \hline
        HU-50 & ~ & X & ~ & ~ & ~ & ~ & ~ & ~ & X \\ \hline
        HU-51 & ~ & ~ & ~ & ~ & ~ & ~ & ~ & X & X \\ \hline
        HU-52 & ~ & ~ & ~ & X & X & ~ & ~ & ~ & X \\ \hline
        HU-53 & ~ & ~ & ~ & ~ & ~ & X & ~ & ~ & X \\ \hline
        HU-54 & ~ & ~ & ~ & ~ & ~ & ~ & ~ & ~ & X \\ \hline
        HU-55 & ~ & ~ & ~ & ~ & ~ & ~ & ~ & ~ & X \\ \hline
        HU-56 & ~ & ~ & ~ & ~ & ~ & ~ & ~ & ~ & X \\ \hline
        HU-57 & ~ & ~ & ~ & ~ & ~ & ~ & ~ & ~ & X \\ \hline
        HU-58 & ~ & ~ & X & ~ & ~ & ~ & ~ & ~ & ~ \\ \hline
        HU-59 & ~ & ~ & X & ~ & ~ & ~ & ~ & ~ & ~ \\ \hline
        HU-60 & ~ & X & X & X & ~ & ~ & ~ & ~ & ~ \\ \hline
        HU-61 & ~ & ~ & X & ~ & ~ & ~ & ~ & ~ & ~ \\ \hline
        HU-62 & ~ & ~ & X & ~ & ~ & ~ & ~ & ~ & ~ \\ \hline
        HU-63 & ~ & ~ & X & ~ & ~ & ~ & ~ & ~ & ~ \\ \hline
        HU-64 & ~ & X & ~ & X & ~ & ~ & ~ & ~ & ~ \\ \hline
        HU-65 & ~ & X & X & ~ & ~ & ~ & ~ & X & ~ \\ \hline
        HU-66 & ~ & ~ & ~ & ~ & X & ~ & ~ & ~ & X \\ \hline
        HU-67 & ~ & ~ & ~ & ~ & ~ & ~ & ~ & ~ & X \\ \hline
        HU-68 & ~ & ~ & ~ & X & ~ & ~ & ~ & ~ & ~ \\ \hline
    \end{tabular}
\end{table}
% !TEX root = ../../proyect.tex

\section{Prototipos de interfaz}\label{sec:prototipos-interfaz}
En esta sección se presentan los prototipos visuales de las pantallas de la aplicación, mostrando la estructura, diseño y funcionalidades de cada pantalla. Ésta representación visual permitirá saber cómo se verá y funcionará la aplicación antes de ser programada. Estos prototipos de interfaz también ayudan a detectar problemas de usabilidad previos a la implementación y a mejorar la experiencia del usuario.

Para la elaboración de los prototipos se ha utilizado la herramienta Figma\footnote{\todo{En todas las URL, poner algo más que solo eso en el pie de página (e.g. "Página web de Figma")}\url{https://www.figma.com/}}, una decisión principalmente motivada por una parte por la existencia de un plan gratuito más que suficiente para las necesidades del proyecto, y por otra parte debido a que ya tenía conocimiento previo de la herramienta. Esto me ha permitido construír prototipos fieles al producto final con facilidad, de modo que la maquetación en CSS se limite a replicar la estética de los mismos:

\subsection{Pantallas de la aplicación}

\todo{Incluír capturas de pantalla de figma con pie de foto diciendo de qué es cada pantalla}
% !TEX root = ../../proyect.tex

\section{Modelo de la base de datos}\label{sec:modelo-de-datos}

En esta sección se abordará la definición de la base de datos relacional que se utilizará en la aplicación, con el objetivo de garantizar su eficiencia, consistencia y escalabilidad. Para ello, se priorizará que la definición cumpla la tercera forma normal, es decir, normalizando en la medida de lo posible los datos y eliminar las dependencias transitivas entre las entidades.
% TODO: reescribir lo de arriba, citar definiciones...

Para entender la tercera forma normal, es importante entender las dos formas normales previas. La primera forma normal (1FN) establece que cada tabla en la base de datos debe tener valores atómicos, es decir, valores que no se puedan descomponer en partes más pequeñas. La segunda forma normal (2FN) establece que, además de lo anterior, cada columna en una tabla debe depender únicamente de la clave primaria de la tabla.

La tercera forma normal (3FN) va un paso más allá y establece que, además de cumplir con la segunda forma normal, las columnas que no son parte de la clave primaria de la tabla deben depender únicamente de la clave primaria de la tabla y no de otras columnas no clave. Esto significa que todas las dependencias transitivas deben eliminarse para evitar la redundancia y la inconsistencia de datos. En otras palabras, cada hecho o atributo debe estar representado en una única tabla y sin información redundante.

En el contexto de la aplicación, esto implica que la base de datos debe estar diseñada de manera que la información se almacene de manera coherente y lógica, evitando la redundancia y la inconsistencia de datos.
% !TEX root = ../../proyect.tex

\section{Requisitos no funcionales}\label{sec:requisitos-no-funcionales}

En esta sección se establecerán los requisitos no funcionales que deben ser satisfechos para garantizar que la aplicación cumpla con las expectativas del usuario y tenga un desempeño adecuado en el entorno previsto. Dichos requisitos definen los criterios que deben cumplirse para garantizar que la aplicación cumpla con los estándares de calidad, seguridad, rendimiento y escalabilidad requeridos. Estos requisitos determinan cómo la aplicación debe funcionar en términos de fiabilidad, capacidad de respuesta y facilidad de uso, entre otros aspectos. Los requisitos no funcionales de la aplicación se detallan a continuación:

\todo{Añadir algunas cosas del keep notes como RNF}

\begin{enumerate}[label=NFR-\protect\twodigits{\arabic*}:, align=left, leftmargin=*]

\item \textbf{Interfaz de usuario intuitiva}: El sistema deberá proporcionar una interfaz de usuario fácil de usar e intuitiva, que permita a los usuarios realizar sus acciones sin dificultad y que sea atractiva visualmente.

\item \textbf{Alta disponibilidad y rendimiento}: La aplicación debe ser confiable y estar disponible para los usuarios en todo momento, con un tiempo de respuesta rápido para proporcionar una buena experiencia de usuario.

\item \textbf{Seguridad}: El sistema deberá garantizar la seguridad de la información de los usuarios, incluyendo datos de pago y datos personales, mediante la implementación de medidas de seguridad adecuadas y cumpliendo con las regulaciones de protección de datos.

\item \textbf{Integración con redes sociales}: El sistema deberá permitir a los usuarios iniciar sesión y compartir información sobre productos y otros contenidos a través de redes sociales, para aumentar la difusión de la aplicación y mejorar su presencia en línea.

\item \textbf{Integración con sistemas externos}: El sistema deberá permitir la integración con sistemas externos como pasarelas de pago y proveedores de servicios de terceros para mejorar la experiencia de usuario y la eficiencia de la aplicación.

\item \textbf{Sistema de notificaciones push}: El sistema deberá contar con un sistema de notificaciones push que permita a los usuarios recibir alertas inmediatas sobre el estado de sus pedidos, cambios en los precios o disponibilidad de productos, así como otras novedades relevantes de la plataforma.

\item \textbf{Personalización de la experiencia de usuario}: El sistema deberá permitir a los usuarios personalizar su experiencia de usuario, incluyendo la configuración de preferencias de notificaciones.

\item \textbf{Analítica de datos}: El sistema deberá ser capaz de recopilar datos sobre la actividad de los usuarios y el rendimiento de la aplicación bajo previa autorización, con el fin de obtener información valiosa sobre el comportamiento de los usuarios y la eficacia de la aplicación.

\end{enumerate}
%     \chapter{Definici\'on de objetivos}\label{defobjetivos}
	
%     \chapter{An\'alisis de requisitos, dise\~no e implementaci\'on}\label{requisitos}




\section{Dise\~no e implementaci\'on}

%     \chapter{An\'alisis de antecedentes y aportaci\'on realizada}\label{analanteced}
 
%     \chapter{An\'alisis temporal y costes de desarrollo}\label{anatemporal}

\section{An\'alisis temporal}
	
\section{Costes de desarrollo}		
	

	
%     \chapter{Comparación con otras alternativas}\label{alternativas}













	








	
	 





%     \chapter{Pruebas}\label{pruebas}



	



%     \chapter{Manual}\label{manual}



\backmatter

% \chapter{Apéndices}\label{apendices}
%\bibliographystyle{sousa5}

\bibliographystyle{apacite}

\bibliography{pfcbib}

\end{document}
