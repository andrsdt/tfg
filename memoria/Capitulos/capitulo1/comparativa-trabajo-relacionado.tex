% !TEX root = ../../proyect.tex

\section{Tabla comparativa del trabajo relacionado}\label{sec:comparativa-trabajo-relacionado}

Para realizar una comparación detallada de las diferentes páginas web similares que hemos encontrado en nuestra investigación, vamos a establecer una serie de atributos que nos permitan evaluarlas de manera objetiva y coherente. A continuación, definiremos los criterios objetivos que usaremos para evaluar cada atributo:

\begin{itemize}

	\item \textbf{Diseño}: evaluaremos el atractivo visual y la coherencia del diseño de la página, así como la facilidad de uso de la interfaz de usuario. Dotaremos a la web de 2 puntos si el diseño es atractivo y coherente, 1 punto si es atractivo pero no coherente o viceversa o 0 puntos si no es atractivo ni coherente.

	\item \textbf{Usabilidad}: evaluaremos la facilidad de uso de la página y cómo de intuitivas son para un usuario nuevo. Dotaremos a la web de 2 puntos si es fácil de usar y la navegación es intuitiva, 1 punto si la web es fácil de usar pero la navegación no es intuitiva o viceversa o 0 puntos si la web es difícil de usar y la navegación no es intuitiva.

	\item \textbf{Contenido}: evaluaremos la relevancia y actualidad del contenido de la plataforma. Dotaremos a la web de 2 puntos si tiene contenido de calidad, relevante y actualizado, 1 punto si la web tiene contenido de calidad pero no está actualizado o viceversa o 0 puntos si el contenido es de baja calidad y no es relevante ni actualizado.

	\item \textbf{Soporte}: evaluaremos la calidad y el nivel de soporte ofrecido por la página web, incluyendo recursos de ayuda, preguntas frecuentes y atención al cliente. Dotaremos a la web de 2 puntos si tiene una sección dedicada a recursos de ayuda o preguntas frecuentes y una sección de atención al cliente, 1 punto si sólo tiene una de las dos o 0 puntos si no dispone de ningún soporte al usuario.

\end{itemize}

Siguiendo este sistema de puntos podemos asignar a cada página web una puntuación máxima de 2 puntos en cada atributo, y una puntuación total máxima de \textbf{8 puntos} a cada uno de ellos.

Con esto podremos evaluar las diferentes plataformas que hemos analizado durante la investigación en base a de los criterios objetivos que hemos definido, con el fin de ayudarnos a identificar huecos en el mercado que nuestra aplicación pueda cubrir (véase el cuadro \ref{tabla:comparativa}).

\begin{table}[h]
	\centering
	\begin{tabular}{|l|c|c|c|c|c|}
	\hline
	\textbf{Plataforma} & \textbf{Diseño} & \textbf{Usabilidad} & \textbf{Contenido} & \textbf{Soporte} & \textbf{Total} \\ \hline
		Naranjas Del Carmen & 2 & 1 & 1 & 2 & 6 \\ \hline
		CrowdFarming        & 2 & 2 & 2 & 2 & 8 \\ \hline
		Farm To People      & 2 & 2 & 2 & 2 & 8 \\ \hline
		Kusikuy             & 0 & 2 & 0 & 0 & 2 \\ \hline
		Bijak               & 2 & 2 & 1 & 1 & 6 \\ \hline
		Mandi Trades        & 0 & 1 & 0 & 0 & 1 \\ \hline
	\end{tabular}
    \caption{Tabla comparativa de las webs analizadas}
    \label{tabla:comparativa}
\end{table}

Tras comparar las diferentes alternativas podemos concluír en que CrowdFarming y FarmToPeople son las más competentes en el sector, precisamente las dos entre las que encontramos más coincidencias visuales (véase la figura \ref{fig:cf-comparativa-ftp}).

\figura{0.8}{img/crowdfarming/comparativa-ftp}{Comparación lado a lado de CrowdFarming y Farm The People}{fig:cf-comparativa-ftp}{}

Tanto CrowdFarming como Farm To People han resultado ser plataformas populares con un modelo de negocio muy similar y con una base de usuarios consolidada, CrowdFarming en Europa y Farm The People en Norteamérica. Podemos achacar su buen rendimiento a que brindan una experiencia agradable tanto a clientes como a productores en cuanto a interfaz y usabilidad, a que proporcionan una plataforma de atención al cliente competente y a que el contenido de calidad en sus respectivos blogs ha logrado posicionarlas en un buen lugar en los motores de búsqueda.

Por otra parte, Naranjas Del Carmen y Bijak son plataformas bastante competentes pero que podrían mejorar en algunos aspectos. En el caso de Naranjas del Carmen, el diseño de la página web es atractivo pero podría ser más intuitivo: No está clara la finalidad de la página de un primer vistazo como ocurre con su plataforma hermana, CrowdFarming, donde podemos ver alimentos en venta desde la misma página principal. En el caso de Bijak, si bien la web es atractiva y rica en contenido, pero el blog\footnote{\url{https://blog.bijak.in/}}, que se actualizaba con publicaciones semanales en 2022, no ha recibido ninguna actualización en los dos primeros meses de 2023. También le penaliza la ausencia de una sección de preguntas frecuentes, presente en la mayoría de plataformas analizadas. 

En cuanto a las que han sido evaluadas con la peor puntuación, Kusikuy y Mandi Trades, se debe principalmente a la ausencia de página web. En el caso de Kusikuy no disponen ni siquiera de una \textquote{landing page}, siendo nulo su posicionamiento en buscadores. En Mandi Trades no encontramos tampoco página web, ya que tras el cese de actividad de la compañía se quedó libre el dominio que alojaba la web.