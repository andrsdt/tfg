% !TEX root = ../../proyect.tex

\section{Introducción al problema}\label{sec:intro}

% Buscar información más reciente (2021, 2022)
Los productores locales son fundamentales para el desarrollo de la economía local, sin embargo, suelen enfrentarse a importantes dificultades a la hora de vender sus productos. Según el Índice de Precios Percibidos (IPP)\footnote{https://www.mapa.gob.es/es/estadistica/temas/estadisticas-agrarias/economia/precios-percibidos-pagados-salarios/precios-percibidos-por-los-agricultores-y-ganaderos}, los precios que reciben los agricultores y ganaderos por sus productos han bajado un 2,6\% en 2020 respecto al año anterior, mientras que la diferencia entre lo que pagan los consumidores y lo que reciben los productores por los alimentos se ha multiplicado por 5,4 en enero de 2020 según el Índice de Precios en Origen y Destino (IPOD)\footnote{https://www.ocu.org/alimentacion/alimentos/noticias/consumidores-produccion-agraria}. Esto se suma al hecho de que muchos productores locales no tengan acceso a los recursos y la infraestructura necesarios para comercializar sus productos con eficacia.

Otro problema creciente es la complejidad de las cadenas de transporte globalizadas, que las hace especialmente vulnerables a interrupciones como las que hemos visto durante la pandemia, lo que puede tener graves consecuencias para el suministro de alimentos (véase la figura \ref{fig:cadena-tradicional}). En este contexto es importante apoyar la producción local de alimentos y fomentar su distribución a través de canales directos y eficientes.

Además de esto, el sector alimentario español ha tenido que afrontar el reciente aumento de los costes de producción y transporte y las prácticas desleales de venta a pérdidas\footnote{https://revistas.eleconomista.es/agro/2021/octubre/paso-adelante-contra-la-venta-a-perdidas-sera-el-definitivo-XG9282642}, siendo indispensable apostar por la innovación y la digitalización para mejorar su competitividad y sostenibilidad\footnote{https://www.eleconomista.es/retail-consumo/noticias/12164072/02/23/La-alimentacion-achaca-la-inflacion-a-los-costes-y-apuesta-por-innovar.html}. Todo esto hace evidente el problema al que se enfrentan los productores locales y la necesidad de encontrar soluciones para ellos.

\figura{0.75}{img/solucion/cadena-tradicional}{Diagrama de la cadena de suministro tradicional en la agricultura}{fig:cadena-tradicional}{}

El objetivo de este proyecto es analizar las dificultades a las que se enfrentan los productores locales en la venta de sus productos y proponer soluciones para mejorar la situación. Para ello, se ha llevado a cabo un análisis de las soluciones ya existentes para lidiar con este problema. Los resultados de este análisis muestran la necesidad de una mayor promoción de los productos locales y el desarrollo de una plataforma en línea para conectar a los productores con  potenciales consumidores. En este trabajo se presentan soluciones concretas y viables para mejorar la comercialización de productos locales y apoyar a los productores en su lucha por una economía más justa y sostenible mediante la creación de una plataforma web de compraventa.

\section{Descripción de la solución propuesta}\label{sec:descripcion-solucion}

Para abordar los retos a los que se enfrentan los productores locales, proponemos una plataforma que facilite la venta directa de sus productos a los consumidores, sin necesidad de intermediarios (véase la figura \ref{fig:cadena-directa}). Esta plataforma ofrecerá una forma sencilla, cómoda y eficaz de conectar a productores y compradores, lo que permitirá a los productores obtener un precio justo por sus productos y llegar a nuevos clientes, y a los compradores acceder a una mayor variedad de productos locales. 

\figura{0.75}{img/solucion/cadena-moderna}{Diagrama de nuestra propuesta de cadena de suministro directa}{fig:cadena-directa}{}

La aplicación se centrará en la sencillez y facilidad de uso, lo que permitirá a los productores poner a la venta sus productos y a los compradores examinarlos y comprarlos con facilidad. La plataforma permitirá a los productores locales conectar con compradores potenciales de su zona, lo que ayudará a garantizar que los productores puedan obtener un precio justo por sus productos.

En conjunto, nuestra plataforma ofrecerá una solución integral que ayudará a los productores locales a vender sus productos directamente a los consumidores y a hacer crecer sus negocios. En resumen, nuestra aplicación tiene como objetivo crear una conexión directa entre productores y compradores, facilitando la venta de productos locales y fomentando la economía local.