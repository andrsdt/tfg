% !TEX root = ../../proyect.tex

\section{Descripción de objetivos}\label{sec:descripcion-objetivos}

La aplicación tiene como objetivo principal proporcionar una plataforma eficiente y segura para que los productores puedan publicar y vender sus productos, y para que los usuarios puedan buscar y comprar productos de manera fácil y conveniente. Para lograr esto, se han establecido una serie de objetivos que se detallan a continuación:

\begin{enumerate}[label=OBJ-\protect\twodigits{\arabic*}:, align=left, leftmargin=*]
 
\item \textbf{Gestión de usuarios}: El sistema deberá permitir a los usuarios registrarse, iniciar sesión, editar su perfil y cerrar sesión en la aplicación. Además, el sistema debería ser capaz de gestionar los roles de los usuarios (usuario autenticado o productor) y restringir el acceso a ciertas funcionalidades según el rol del usuario.

\item \textbf{Gestión de productos}: El sistema deberá permitir a los productores publicar sus productos, editar la información de sus productos y eliminarlos. Los usuarios deben poder buscar y navegar entre los diferentes productos, así como ver detalles de los productos (precio, descripción, etc.).

\item \textbf{Gestión de pedidos}: El sistema deberá permitir a los usuarios comprar productos y realizar pagos, así como a los productores gestionar y aceptar/rechazar pedidos.

\item \textbf{Gestión de reseñas}: El sistema deberá permitir a los usuarios dejar reseñas y calificaciones para los productos que han adquirido, a modo de retroalimentación a los productores para mejorar la calidad de sus productos.

\item \textbf{Gestión de denuncias}: El sistema deberá permitir a los usuarios enviar reportes o denuncias a productos que no cumplan con las reglas de la aplicación, así como reportar problemas con sus pedidos, para asegurar la satisfacción de los clientes y poder solucionar problemas con los productores.

\item \textbf{Gestión de notificaciones}: El sistema deberá permitir a los usuarios autenticados recibir notificaciones relevantes relacionadas con su perfil, interacciones y eventos importantes en la aplicación. Los usuarios deben tener la capacidad de configurar sus preferencias de notificación y elegir los canales de comunicación (in-app, push, correo electrónico, SMS) a través de los cuales desean recibir las notificaciones.

\item \textbf{Gestión de conversaciones}: El sistema deberá permitir a los usuarios intercambiar mensajes con un sistema de chat y mensajería instantánea que facilite la comunicación entre los usuarios, permitiendo enviar y recibir mensajes de manera rápida y efectiva.

\end{enumerate}