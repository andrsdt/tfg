% !TEX root = ../../proyect.tex

\section{Historias de usuario}\label{sec:historias-usuario}

En esta sección, se presentarán las historias de usuario de la aplicación que se va a desarrollar. Las historias de usuario son descripciones detalladas de funcionalidades que deben ser implementadas en la aplicación desde el punto de vista del usuario. Cada historia de usuario representa una necesidad específica del usuario que la aplicación debe satisfacer. Estas historias se utilizarán como base para el desarrollo de la aplicación y guiarán todo el proceso de diseño y desarrollo.

\begin{enumerate}[HU-1:]

\subsection{Administradores}

\item \textbf{Gestionar publicaciones}: Como administrador, quiero poder eliminar publicaciones inapropiadas o spam en la plataforma para mantener la calidad de la experiencia de usuario.

\item \textbf{Gestionar categorías}: Como administrador, quiero poder agregar nuevas categorías de productos para que los productores puedan publicar productos más específicos.

\item \textbf{Gestionar pedidos}: Como administrador, quiero ver una lista de todos los pedidos realizados en la plataforma, para poder asegurarme de que se están cumpliendo correctamente.

\item \textbf{Gestionar usuarios}: Como administrador, quiero poder ver una lista de todos los usuarios registrados en la plataforma, para poder verificar su identidad y asegurarme de que cumplen con los requisitos para ser usuarios de la plataforma.

\item \textbf{Gestionar quejas y reclamaciones}: Como administrador, quiero poder gestionar las quejas y reclamaciones de los usuarios, para poder resolver los problemas de manera efectiva y garantizar la satisfacción del usuario.

\item \textbf{Crear promociones}: Como administrador, quiero poder crear promociones para productos específicos en la plataforma, para fomentar la venta de ciertos productos y mejorar la rentabilidad de la plataforma.

\item \textbf{Ver estadísticas y análisis de la plataforma}: Como administrador, quiero poder ver estadísticas y análisis de la plataforma, para poder tomar decisiones estratégicas y mejorar la rentabilidad de la plataforma.

\subsection{Productores}

\item \textbf{Crear publicaciones}: Como productor, quiero poder publicar mis productos en venta para que los consumidores locales puedan ver lo que tengo disponible y ponerse en contacto conmigo para realizar una compra en persona.

\item \textbf{Gestionar ventas}: Como productor, quiero poder gestionar las ventas realizadas y ver el historial de ventas para llevar un control de mis ingresos.

\item \textbf{Gestionar disponibilidad}: Como productor, quiero poder actualizar la disponibilidad de mis productos en tiempo real para evitar que los consumidores se interesen por productos que ya no tengo disponibles.

\item \textbf{Gestionar reservas}: Como productor, quiero poder marcar mis publicaciones como reservadas o vendidas para evitar confusiones o malentendidos con los consumidores.

\item \textbf{Publicar una cosecha en venta}: Como productor, quiero poder publicar en la plataforma la cosecha que tengo disponible para su venta, incluyendo información como el tipo de producto, cantidad, precio y fecha de recolección, para que los consumidores interesados puedan contactarme para realizar el intercambio en persona.

\item \textbf{Actualizar la información de una cosecha en venta}: Como productor, quiero poder editar la información de una cosecha que ya he publicado en la plataforma, para corregir errores o actualizar la información de la disponibilidad del producto.

\item \textbf{Eliminar una cosecha en venta}: Como productor, quiero poder eliminar una cosecha que ya no está disponible para la venta, para evitar que los consumidores intenten contactarme para realizar una compra.

\item \textbf{Recibir notificaciones de nuevos pedidos}: Como productor, quiero recibir una notificación en la plataforma o por correo electrónico cuando un consumidor realiza un nuevo pedido de alguno de mis productos, para que pueda contactarlo y coordinar la entrega.

\item \textbf{Visualizar el historial de pedidos}: Como productor, quiero poder ver el historial de pedidos realizados por los consumidores a través de la plataforma, para poder llevar un registro y mejorar mi planificación de la producción futura.

\subsection{Consumidores}

\item \textbf{Ver lista de productos en venta}: Como consumidor, quiero poder ver una lista de productos destacados venta para poder decidir qué productos comprar.

\item \textbf{Buscar productos}: Como consumidor, quiero poder buscar mediante un buscador productos locales en mi zona para poder apoyar a los productores locales y comprar productos frescos y de calidad.

\item \textbf{Filtrar lista de productos por categoría}: Como consumidor, quiero poder filtrar la lista de productos en venta por categoría (por ejemplo, frutas, verduras, carne, lácteos...) para poder encontrar productos específicos.

\item \textbf{Ver detalles de un producto}: Como consumidor, quiero poder ver los detalles de un producto en particular (descripción, precio, ubicación del productor) para poder tomar una decisión informada sobre si quiero comprar ese producto o no.

\item \textbf{Agregar producto al carrito de compras}: Como consumidor, quiero poder agregar productos al carrito de compras para poder comprar varios productos a la vez.

\item \textbf{Editar cantidad de producto en el carrito de compras}: Como consumidor, quiero poder editar la cantidad de un producto en el carrito de compras para poder ajustar la cantidad de productos que voy a comprar.

\item \textbf{Eliminar producto del carrito de compras}: Como consumidor, quiero poder eliminar productos del carrito de compras para poder comprar solo los productos que realmente quiero.

\item \textbf{Realizar el pago de los productos en el carrito de compras}: Como consumidor, quiero poder realizar el pago de los productos en el carrito de compras para poder finalizar la compra y recibir los productos.

\item \textbf{Ver historial de compras}: Como consumidor, quiero poder ver mi historial de compras para poder recordar qué productos he comprado anteriormente.

\item \textbf{Contactar con productores}: Como consumidor, quiero poder contactar directamente con el agricultor para acordar un punto de encuentro y realizar la compra en persona.

\item \textbf{Ubicación de productores}: Como consumidor, quiero poder ver la ubicación de los productores en un mapa para poder encontrar el punto de encuentro más cercano.

\item \textbf{Gestionar reseñas}: Como consumidor, quiero poder dejar una reseña o comentario en la publicación de un productor para ayudar a otros consumidores a tomar una decisión informada.

\end{enumerate}