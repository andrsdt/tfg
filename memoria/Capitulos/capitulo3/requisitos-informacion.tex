% !TEX root = ../../proyect.tex

\section{Requisitos de información}\label{sec:requisitos-informacion}

En esta sección se definirán los atributos que serán utilizados por el sistema, sirviendo como punto de partida para el desarrollo de la base de datos y la interfaz de usuario. En el caso de la aplicación, los principales objetos que estarán presentes son los siguientes:

\begin{enumerate}[label=IRQ-\protect\twodigits{\arabic*}:, align=left, leftmargin=*]	

    \item \textbf{Administrador}: El sistema deberá almacenar la información correspondiente a los administradores del sistema. En concreto:
    \begin{itemize}
        \item \texttt{integer}. Identificador de administrador
        \item \texttt{string}. Correo electrónico
        \item \texttt{string}. Contraseña
    \end{itemize}
    
    \item \textbf{Usuario}: El sistema deberá almacenar la información correspondiente a los usuarios del sistema. En concreto:
    \begin{itemize}
		\item \texttt{integer}. Identificador de usuario
        \item \texttt{date}. Fecha de creación
        \item \texttt{date}. Fecha de última modificación
        \item \texttt{string}. Correo electrónico
		\item \texttt{string}. Contraseña
        \item \texttt{string}. Nombre
        \item \texttt{string}. Apellido(s)
		\item \texttt{string}. Número de teléfono
		\item \texttt{point}. Ubicación
		\item \texttt{image}. Foto de perfil
    \end{itemize}

    \item \textbf{Productor}: El sistema deberá almacenar la información correspondiente a los productores del sistema. En concreto:
    \begin{itemize}
		\item \texttt{integer}. Identificador de productor
        \item \texttt{string}. Biografía
    \end{itemize}
    
    \item \textbf{Publicación}: El sistema deberá almacenar la información correspondiente a cada producto publicado por un productor. En concreto:
    \begin{itemize}
        \item \texttt{integer}. Identificador de publicación
        \item \texttt{date}. Fecha de creación
        \item \texttt{date}. Fecha de última modificación
        \item \texttt{string}. Título
        \item \texttt{string}. Descripción
        \item \texttt{string}. Unidad (al peso o unitario)
        \item \texttt{integer}. Precio por unidad (en céntimos)
        \item \texttt{integer}. Stock disponible
        \item \texttt{list(image)}. Lista de imágenes
        \item \texttt{list(string)}. Lista de alérgenos
        \item \texttt{list(string)}. Lista de características
        \item \texttt{boolean}. Indicador de activo
    \end{itemize}

    \item \textbf{Pedido}: El sistema deberá almacenar la información correspondiente a los pedidos realizados por los consumidores. En concreto:
    \begin{itemize}
        \item \texttt{integer}. Identificador de pedido
        \item \texttt{date}. Fecha de creación
        \item \texttt{Publicación}. Publicación sobre la que se realiza el pedido
        \item \texttt{Usuario}. Usuario que realiza el pedido
        \item \texttt{float}. Cantidad de producto adquirida
        \item \texttt{float}. Precio total de la venta
    \end{itemize}

    \item \textbf{Reseña}: El sistema deberá almacenar la información correspondiente a las calificaciones realizadas a los productores por parte de los compradores de sus publicaciones. En concreto:
    \begin{itemize}
        \item \texttt{integer}. Identificador de reseña
        \item \texttt{date}. Fecha de creación
        \item \texttt{date}. Fecha de última modificación
        \item \texttt{integer}. Puntuación (1-5 estrellas)
        \item \texttt{string}. Comentario
        \item \texttt{Usuario}. Usuario que realiza la reseña
        \item \texttt{Publicación}. Publicación sobre la que se realiza la reseña
    \end{itemize}

    \item \textbf{Denuncia}: El sistema deberá almacenar la información correspondiente a las denuncias o reportes de problemas emitidos por los usuarios, referentes bien a un productor o bien a un pedido. En concreto:
    \begin{itemize}
        \item \texttt{integer}. Identificador de denuncia
        \item \texttt{date}. Fecha de creación
        \item \texttt{date}. Fecha de última modificación
        \item \texttt{Usuario}. Usuario emisor de la denuncia
        \item \texttt{Usuario}. Usuario receptor de la denuncia
        \item \texttt{string}. Descripción del problema
        \item \texttt{boolean}. Indicador de denuncia ya resuelta
        \item \texttt{Pedido}. Referencia al pedido denunciado (opcional)
    \end{itemize}

    \item \textbf{Conversación}: El sistema deberá almacenar la información correspondiente a cada conversación que mantengan un productor y un consumidor en lo referente a un producto. En concreto:
    \begin{itemize}
        \item \texttt{integer}. Identificador de conversación
        \item \texttt{date}. Fecha de creación
        \item \texttt{date}. Fecha de última modificación
        \item \texttt{Usuario}. Comprador asociado a la conversación
        \item \texttt{Publicación}. Publicación sobre la cual trata la conversación
    \end{itemize}

    \item \textbf{Mensaje}: El sistema deberá almacenar la información correspondiente a cada mensaje enviado por un productor o un consumidor en una conversación. En concreto:
    \begin{itemize}
        \item \texttt{integer}. Identificador de mensaje
        \item \texttt{date}. Fecha de creación
        \item \texttt{date}. Fecha de última modificación
        \item \texttt{string}. Tipo de mensaje
        \item \texttt{string}. Texto
        \item \texttt{string}. Datos (Campo JSON flexible para mensajes más sofisticados que texto plano)
        \item \texttt{Usuario}. Remitente
        \item \texttt{Usuario}. Destinatario
        \item \texttt{boolean}. Indicador de leído por el destinatario
    \end{itemize}

    
    \item \textbf{Notificación}: El sistema deberá almacenar la información correspondiente a cada notificación emitida a un usuario. En concreto:
    \begin{itemize}
        \item \texttt{integer}. Identificador de notificación
        \item \texttt{date}. Fecha de creación
        \item \texttt{date}. Fecha de última modificación
        \item \texttt{string}. Tipo de notificación
        \item \texttt{string}. Campo flexible de datos concretos de la notificación
        \item \texttt{Usuario}. Destinatario
        \item \texttt{boolean}. Indicador de leído por el destinatario
    \end{itemize}

\end{enumerate}

\figura{0.5}{img/diagramas/irq}{Diagrama de las relaciones entre los requisitos de información}{fig:diagrama-irq}{}

\todo{Esta es la primera versión, exportar el nuevo de PlantUML}