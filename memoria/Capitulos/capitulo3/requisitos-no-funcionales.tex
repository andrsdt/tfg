% !TEX root = ../../proyect.tex

\section{Requisitos no funcionales}\label{sec:requisitos-no-funcionales}

En esta sección se establecerán los requisitos no funcionales que deben ser satisfechos para garantizar que la aplicación cumpla con las expectativas del usuario y tenga un desempeño adecuado en el entorno previsto. Dichos requisitos definen los criterios que deben cumplirse para garantizar que la aplicación cumpla con los estándares de calidad, seguridad, rendimiento y escalabilidad requeridos. Estos requisitos determinan cómo la aplicación debe funcionar en términos de fiabilidad, capacidad de respuesta y facilidad de uso, entre otros aspectos. Los requisitos no funcionales de la aplicación se detallan a continuación:

\todo{Añadir algunas cosas del keep notes como RNF}

\begin{enumerate}[label=NFR-\protect\twodigits{\arabic*}:, align=left, leftmargin=*]

\item \textbf{Interfaz de usuario intuitiva}: El sistema deberá proporcionar una interfaz de usuario fácil de usar e intuitiva, que permita a los usuarios realizar sus acciones sin dificultad y que sea atractiva visualmente.

\item \textbf{Alta disponibilidad y rendimiento}: La aplicación debe ser confiable y estar disponible para los usuarios en todo momento, con un tiempo de respuesta rápido para proporcionar una buena experiencia de usuario.

\item \textbf{Seguridad}: El sistema deberá garantizar la seguridad de la información de los usuarios, incluyendo datos de pago y datos personales, mediante la implementación de medidas de seguridad adecuadas y cumpliendo con las regulaciones de protección de datos.

\item \textbf{Integración con redes sociales}: El sistema deberá permitir a los usuarios iniciar sesión y compartir información sobre productos y otros contenidos a través de redes sociales, para aumentar la difusión de la aplicación y mejorar su presencia en línea.

\item \textbf{Integración con sistemas externos}: El sistema deberá permitir la integración con sistemas externos como pasarelas de pago y proveedores de servicios de terceros para mejorar la experiencia de usuario y la eficiencia de la aplicación.

\item \textbf{Sistema de notificaciones push}: El sistema deberá contar con un sistema de notificaciones push que permita a los usuarios recibir alertas inmediatas sobre el estado de sus pedidos, cambios en los precios o disponibilidad de productos, así como otras novedades relevantes de la plataforma.

\item \textbf{Personalización de la experiencia de usuario}: El sistema deberá permitir a los usuarios personalizar su experiencia de usuario, incluyendo la configuración de preferencias de notificaciones.

\item \textbf{Analítica de datos}: El sistema deberá ser capaz de recopilar datos sobre la actividad de los usuarios y el rendimiento de la aplicación bajo previa autorización, con el fin de obtener información valiosa sobre el comportamiento de los usuarios y la eficacia de la aplicación.

\end{enumerate}