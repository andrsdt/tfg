% !TEX root = ../../proyect.tex

\section{Casos de uso}\label{sec:casos-uso}
En esta sección, se presentan los casos de uso para la plataforma en línea de venta de productos. Estos casos de uso se han diseñado para permitir a los usuarios comprar y vender productos de forma eficiente y efectiva. Los casos de uso proporcionan una descripción detallada de las diferentes funcionalidades de la plataforma, y se han organizado en función del tipo de usuario que interactúa con el sistema.

Estos casos de uso describen los diferentes escenarios de interacción entre los usuarios (actores) y el sistema. Cada caso de uso describe una acción específica que el usuario puede realizar y cómo el sistema responde a ella, incluyendo los pasos que el usuario debe seguir para completar la acción, y se detallan posibles excepciones y flujos alternativos que el sistema puede tomar.

\begin{enumerate}[CU-1:]
\item{Registro de usuario}
\begin{itemize}
	\item \textbf{Actor:} Usuario
	\item \textbf{Descripción:} Permite al usuario crear una cuenta en la aplicación.
	\item \textbf{Flujo básico de eventos:}
	\begin{enumerate}[1:]
		\item El usuario accede a la página de registro.
		\item El sistema muestra un formulario de registro con los campos obligatorios: nombre, correo electrónico y contraseña.
		\item El usuario rellena el formulario con sus datos personales y confirma su registro.
		\item El sistema verifica que los datos sean válidos, crea una cuenta para el usuario y lo redirige a la página de inicio de sesión
	\end{enumerate}
	\item \textbf{Flujos alternativos:}
		\begin{itemize}
		\item[4a.] Si el correo electrónico ya está registrado, el sistema muestra un mensaje de error y pide al usuario que proporcione otro correo electrónico.
		\end{itemize}
\end{itemize}

\item{Inicio de sesión}
\begin{itemize}
	\item \textbf{Actor:} Usuario registrado
	\item \textbf{Descripción:} Permite al usuario iniciar sesión en su cuenta de la aplicación.
	\item \textbf{Flujo básico de eventos:}
	\begin{enumerate}[1:]
		\item El usuario accede a la página de inicio de sesión.
		\item El sistema muestra un formulario para introducir el correo electrónico y la contraseña del usuario.
		\item El usuario introduce su correo electrónico y su contraseña.
		\item El sistema comprueba que los datos son correctos, concede el acceso a la cuenta del usuario y lo redirige a su página de inicio.
		\end{enumerate}
	\item \textbf{Flujos alternativos:}
		\begin{itemize}
		\item[4a.] Si los datos son incorrectos, el sistema muestra un mensaje de error y solicita que se vuelvan a introducir.
		\end{itemize}
	\item \textbf{Excepciones:}
		\begin{itemize}
		\item[4a.] Si el sistema detecta algún problema técnico, muestra un mensaje de error y pide al usuario que lo intente más tarde.
		\end{itemize}
\end{itemize}

\item{Edición de perfil de usuario}
\begin{itemize}
	\item \textbf{Actor:} Usuario
	\item \textbf{Descripción:} Permite al usuario editar su información personal y de contacto.
	\item \textbf{Flujo básico de eventos:}
	\begin{enumerate}[1:]
		\item El usuario accede a la sección de \textquote{Editar perfil} en su cuenta.
		\item El sistema muestra la información personal y de contacto del usuario.
		\item El usuario edita los campos que desee modificar y confirma los cambios.
		\item El sistema actualiza la información del perfil del usuario.
		\end{enumerate}
	\item \textbf{Flujos alternativos:}
		\begin{itemize}
		\item[3a.] Si el usuario desea cancelar la edición de su perfil, puede seleccionar la opción de \textquote{Cancelar} y el sistema no realizará cambios en la información del perfil.
		\item[4a.] Si el usuario no completa algún campo obligatorio, el sistema mostrará un mensaje de error y no permitirá la confirmación de los cambios hasta que el campo sea completado correctamente.
		\end{itemize}
\end{itemize}

\item{Gestión de usuarios}
\begin{itemize}
	\item \textbf{Actor:} Administrador
	\item \textbf{Descripción:} Permite al administrador gestionar la información de los usuarios registrados en la plataforma.
	\item \textbf{Flujo básico de eventos:}
	\begin{enumerate}[1:]
		\item El administrador accede a la sección de gestión de usuarios.
		\item El sistema muestra una lista de todos los usuarios registrados en la plataforma.
		\item El administrador selecciona un usuario de la lista para ver su información.
		\item El sistema muestra la información del usuario seleccionado, incluyendo su nombre, correo electrónico, fecha de registro y historial de compras en el caso de un consumidor (o ventas y productos publicados en el caso de un productor).
		\item El administrador tiene la opción de editar la información del usuario o de eliminarlo de la plataforma.
		\item El sistema actualiza la lista de usuarios después de realizar cualquier cambio.
		\end{enumerate}
	\item \textbf{Flujos alternativos:}
	\begin{itemize}
		\item [5a.] Si se elige bloquear o eliminar un usuario, el sistema solicita confirmación antes de llevar a cabo la acción.
		\end{itemize}
\end{itemize}

\item{Gestión de productos}
\begin{itemize}
	\item \textbf{Actor:} Administrador
	\item \textbf{Descripción:} Permite al administrador gestionar los productos de la plataforma.
	\item \textbf{Flujo básico de eventos:}
	\begin{enumerate}[1:]
		\item El administrador accede al panel de administración de productos.
		\item El sistema muestra la lista de productos disponibles en la plataforma.
		\item El administrador puede modificar información de los productos existentes (nombre, descripción, precio, categoría, etc.) o eliminarlos de la plataforma.
		\item El administrador confirma la acción correspondiente.
		\item El sistema registra la acción y muestra un mensaje de confirmación.
		\end{enumerate}
	\item \textbf{Flujos alternativos:}
		\begin{itemize}
		\item [2a.] Si no hay productos registrados en la plataforma, el sistema muestra un mensaje indicando que no hay productos disponibles.
		\item [3a.] Si el administrador intenta realizar una acción no permitida (por ejemplo, eliminar un producto que tiene órdenes pendientes o que no le pertenece), el sistema muestra un mensaje de error indicando que la acción no es posible.
		\end{itemize}
\end{itemize}

\item{Gestión de pedidos}
\begin{itemize}
	\item \textbf{Actor:} Administrador
	\item \textbf{Descripción:} Permite al administrador gestionar los pedidos recibidos en la plataforma.
	\item \textbf{Flujo básico de eventos:}
	\begin{enumerate}[1:]
		\item El administrador accede a la sección de pedidos en su panel de control.
		\item El sistema muestra una lista con los pedidos recibidos, indicando el número de pedido, la fecha, el comprador y el estado del pedido.
		\item El administrador selecciona el pedido que desea gestionar.
		\item El sistema muestra la información detallada del pedido, incluyendo el consumidor y el productor involucrados, los productos comprados y el método de pago utilizado.
		\item El administrador modifica los detalles del pedido según corresponda.
		\item El sistema actualiza el estado del pedido en la base de datos y notifica al comprador del cambio de estado.
		\item El administrador puede añadir notas internas al pedido si lo considera necesario.
		\item El sistema registra las notas internas en la base de datos y las presenta en la sección correspondiente del pedido.
		\end{enumerate}
	\item \textbf{Flujos alternativos:}
		\begin{itemize}
		\item [5a.] Si el administrador detecta algún problema con el pago o la dirección de envío, puede contactar al comprador para solicitar la corrección de los datos.
		\item [6a.] Si el administrador cancela el pedido, el sistema emite un reembolso al comprador si corresponde y actualiza el inventario de los productos en la base de datos.
		\end{itemize}
\end{itemize}

\item{Búsqueda de productos}
\begin{itemize}
	\item \textbf{Actor:} Consumidor
	\item \textbf{Descripción:} Permite al consumidor buscar productos en la aplicación.
	\item \textbf{Flujo básico de eventos:}
	\begin{enumerate}[1:]
		\item El consumidor accede a la página de búsqueda de productos.
		\item El sistema presenta una barra de búsqueda.
		\item El consumidor introduce el término de búsqueda.
		\item El sistema muestra una lista de palabras clave que coinciden con el término de búsqueda
		\item El consumidor hace clic en el botón \textquote{buscar}
		\item El sistema muestra una lista de productos incluyendo su primera imagen, título, precio y distancia
		\item El consumidor hace clic en el producto para acceder a su página de información.
		\item El sistema muestra la información del producto, incluyendo sus imágenes, título, descripción, categoría, precio, disponibilidad y ubicación.
		\end{enumerate}
	\item \textbf{Flujos alternativos:}
	\begin{enumerate}
		\item[6a.] Si no se encuentra ningún producto que coincida con el término de búsqueda, el sistema muestra un mensaje indicando que no se han encontrado resultados.
		\end{enumerate}
	\item \textbf{Excepciones:}
	\begin{enumerate}
		\item[6a.] Si el término de búsqueda introducido por el consumidor contiene caracteres no alfanuméricos, el sistema muestra un mensaje de error indicando que el término de búsqueda es incorrecto.
		\end{enumerate}
\end{itemize}

\item{Publicación de productos}
\begin{itemize}
	\item \textbf{Actor:} Productor
	\item \textbf{Descripción:} Permite al productor publicar un nuevo producto en la plataforma.
	\item \textbf{Flujo básico de eventos:}
	\begin{enumerate}[1:]
		\item El productor accede al apartado de \textquote{Publicar producto} en su perfil.
		\item El sistema muestra un formulario con los campos para introducir la información del producto: imágenes, título, descripción, categoría, precio, disponibilidad y ubicación
		\item El productor completa los campos y envía el formulario.
		\item El sistema valida la información y registra el nuevo producto en la base de datos y muestra una confirmación de que el producto ha sido publicado con éxito.
		\end{enumerate}
	\item \textbf{Flujos alternativos:}
		\begin{itemize}
		\item[2a.] El productor decide no publicar el producto: Si el productor decide no publicar el producto, puede cancelar el proceso y volver al menú principal.
		\item[3a.] El formulario contiene errores: Si el formulario contiene errores o información incompleta, el sistema muestra una notificación de error y pide al productor que corrija los campos necesarios.
		\end{itemize}
\end{itemize}

\item{Compra de productos}
\begin{itemize}
	\item \textbf{Actor:} Consumidor
	\item \textbf{Descripción:} Permite al consumidor adquirir un producto específico.
	\item \textbf{Precondiciones:} El consumidor ha iniciado sesión y ha buscado y seleccionado el producto que desea adquirir.
	\item \textbf{Flujo básico de eventos:}
	\begin{enumerate}[1:]
		\item El consumidor accede a la página del producto que desea adquirir.
		\item El sistema muestra la información del producto, incluyendo su precio y la opción de adquirirlo.
		\item El consumidor selecciona la opción de adquirir el producto.
		\item El sistema muestra al consumidor la confirmación de la compra y el precio total.
		\item El consumidor confirma la compra.
		\item El sistema registra la compra, la hace llegar al productor correspondiente y presenta al consumidor la información de contacto del productor.
		\end{enumerate}
	\item \textbf{Flujos alternativos:}
	\begin{enumerate}
		\item [2a.] Si el sistema detecta que el producto ya ha sido vendido, se mostrará un mensaje de que el producto ya no está disponible.
		\item [4a.] Si el consumidor cambia de opinión o detecta un error, puede cancelar la compra en lugar de confirmarla.
		\item [7a.] Si el consumidor no puede contactar con el productor en un plazo después de la compra, se le dará la opción de cancelar la compra y recibir un reembolso en el caso de haber pagado mediante la aplicación
		\end{enumerate}
	\item \textbf{Excepciones:}
	\begin{enumerate}
		\item[5a.] Si el consumidor no ha iniciado sesión, se le solicitará que inicie sesión antes de poder adquirir el producto.
		\item[6a.] Si el sistema detecta que el consumidor ha proporcionado información de pago incorrecta o no válida, se le solicitará que proporcione información de pago correcta antes de poder confirmar la compra.
		\end{enumerate}
\end{itemize}

\item{Cancelación de pedido}
\begin{itemize}
\item \textbf{Actor:} Consumidor
\item \textbf{Descripción:} Permite al consumidor cancelar un pedido antes de su confirmación por parte del productor
\item \textbf{Flujo básico de eventos:}
	\begin{enumerate}[1:]
\item El consumidor accede a su historial de pedidos.
\item El sistema muestra una lista de los pedidos realizados por el consumidor.
\item El consumidor selecciona el pedido que desea cancelar.
\item El sistema muestra la información del pedido, incluyendo la opción de cancelarlo.
\item El consumidor confirma la cancelación del pedido.
\item El sistema cancela el pedido y emite un reembolso al consumidor si corresponde.
\end{enumerate}
\item \textbf{Flujos alternativos:}
\begin{enumerate}
\item [4a.] Si el pedido ya ha sido confirmado por el productor, el sistema notifica al consumidor que no es posible cancelar el pedido.
\item [5a.] Si se ha realizado un pago con tarjeta de crédito, el reembolso puede tardar varios días hábiles en aparecer en la cuenta del consumidor.
\end{enumerate}
\end{itemize}

\item{Consulta de pedidos recibidos}
\begin{itemize}
	\item \textbf{Actor:} Productor
	\item \textbf{Descripción:} Permite al productor ver los pedidos recibidos por sus productos.
	\item \textbf{Flujo básico de eventos:}
	\begin{enumerate}[1:]
		\item El productor accede a la sección de \textquote{Pedidos recibidos} en su perfil de usuario.
		\item El sistema muestra una lista de los pedidos recibidos, ordenados por fecha de solicitud.
		\item El productor selecciona un pedido de la lista para ver más detalles.
		\item El sistema muestra la información del pedido, incluyendo el nombre y la dirección del comprador, los productos solicitados, el precio total y el estado actual del pedido.
		\end{enumerate}
	\item \textbf{Flujos alternativos:}
		\begin{itemize}
		\item [2a.] Si el productor no tiene ningún pedido recibido, el sistema muestra un mensaje indicando que no hay pedidos disponibles.
		\item [3a.] Si el productor selecciona un pedido que ha sido cancelado o finalizado, el sistema muestra la información del pedido pero indica que ya no puede ser modificado.
		\end{itemize}
\end{itemize}

\item{Rechazo de pedido}
\begin{itemize}
\item \textbf{Actor:} Productor
\item \textbf{Descripción:} Permite al productor rechazar un pedido recibido del consumidor.
\item \textbf{Precondición:} El productor ha recibido un pedido del consumidor.
\item \textbf{Flujo básico de eventos:}
\begin{enumerate}[1:]
\item El productor accede a la lista de pedidos recibidos.
\item El sistema muestra la lista de pedidos recibidos.
\item El productor selecciona el pedido que desea rechazar.
\item El sistema muestra la información del pedido, incluyendo la opción de rechazarlo.
\item El productor selecciona la opción de rechazar el pedido.
\item El sistema muestra un mensaje de confirmación para rechazar el pedido y da la posibilidad al productor de añadir una nota para el consumidor.
\item El productor confirma la acción.
\item El sistema registra la cancelación del pedido y notifica al comprador.
\end{enumerate}
\item \textbf{Flujos alternativos:}
\begin{itemize}
\item[5a.] Si el pedido ya ha sido entregado al comprador, el productor no podrá rechazarlo.
\item[7a.] Si el productor decide no rechazar el pedido, éste sigue su curso.
\end{itemize}
\end{itemize}
\end{enumerate}
